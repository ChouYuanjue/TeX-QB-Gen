\documentclass[12pt]{book}
\usepackage{amsmath,amssymb}
\usepackage{ctex}
\begin{document}
\tableofcontents
\newpage
\chapter{Algebra Test}
\section{Algebra Q1}
\section*{Exercise}
问题 1. 设 $G$ 为对称群 $S_n$ 的一个阿贝尔子群,$p_1, \dots, p_k$ 为 $|G|$ 的所有素因数。证明 $p_1 + \dots + p_k \le n$。



\subsection*{Solution}
证明:假设,为了矛盾,$p_1 + \dots + p_k > n$。由于 $p_i$ 是一个能整除 $|G|$ 的素数,群 $G$ 必须有一个元素的阶为 $p_i$。因此,对于每个 $i$,$G$ 包含一个元素 $\sigma_i$,其循环分解为 $p_i$ 个循环的乘积。对于每个 $j$,我们用 $M_j$ 表示 $J_n = \{1, \dots, n\}$ 中不被 $\sigma_j$ 固定的元素集。由于 $G$ 是阿贝尔群,$M_i$ 和 $M_j$ 对于 $i \ne j$ 是不相交的。事实 $|M_j| \ge p_j$ 意味着
\[|J_n| \ge |M_1| + \dots + |M_k| \ge p_1 + \dots + p_k > n;\]
这与事实相矛盾。因此,$p_1 + \dots + p_k \le n$。
\newpage
\section{Algebra Q2}
\section*{Exercise}
问题 1. 设 $G$ 为对称群 $S_n$ 的一个阿贝尔子群,$p_1, \dots, p_k$ 为 $|G|$ 的所有素因数。证明 $p_1 + \dots + p_k \le n$。



\subsection*{Solution}
解决方案:假设,通过反证,$p_1 + \dots + p_k > n$。由于$p_i$是整除$|G|$的素数,群$G$必须有一个元素的阶为$p_i$。因此,对于每个$i$,$G$包含一个元素$\sigma_i$,其循环类型分解是$p_i$-循环的乘积。对于每个$j$,用$M_j$表示$J_n = \{1, \dots, n\}$中不被$\sigma_j$固定元素的集合。由于$G$是阿贝尔群,$M_i$和$M_j$对于$i \ne j$是分离的。事实$|M_j| \ge p_j$意味着
\[ |J_n| \ge |M_1| + \dots + |M_k| \ge p_1 + \dots + p_k > n; \]
这是一个矛盾。因此,$p_1 + \dots + p_k \le n$。
\newpage
\section{Algebra Q3}
\section*{Exercise}
设 $G$ 为一个有限群,作用于一个有限集合 $S$ 上。对于固定的 $x \in G$,定义 $f(x)$ 为集合 $S$ 中元素 $s$ 的数量,使得 $xs = s$。证明 $G$ 在 $S$ 上的轨道数量等于
\[
\frac{1}{|G|} \sum_{x \in G} f(x).
\]



\subsection*{Solution}
考虑集合 $A = \{(x, s) \in G \times S : xs = s\}$. 我们将 $s \in S$ 的轨道记为 $C_s$。注意到 $|A| = \sum_{x \in G} f(x)$。另一方面,根据轨道-稳定子定理,
\[
|A| = \sum_{s \in S} |\text{Stab}(s)| = \sum_{s \in S} \frac{|G|}{|C_s|} = |G| \sum_{s \in S} \frac{1}{|C_s|}.
\]
由于 $\sum_{s \in S} \frac{1}{|C_s|}$ 等于轨道的数量,因此所需的公式得以证明。
\newpage
\section{Algebra Q4}
\section*{Exercise}
问题3. 如果同一种颜色的珍珠是无法区分的,那么可以设计出多少种由17颗黑白珍珠组成的项链?



\subsection*{Solution}
解决方案:令 $S$ 表示可以用黑白珠子设计的所有 17 颗珠子(带锁)的项链集合。由于 $|S|=2^{17}$。二面体群 $G=D_{2n}$ 以明显的方式作用于 $S$。由于我们感兴趣的是没有锁的项链,对于我们来说,两个项链相同当且仅当它们在 $G$ 对 $S$ 的作用下处于同一个轨道。因此,我们只需要计算 $S$ 在此作用下的轨道数量。$G$ 的恒等式固定了 $2^{17}$ 个项链;每个 $16$ 个非平凡旋转固定了 $2$ 个项链(这是因为 $17$ 是质数);每个 $17$ 个反转固定了 $2^{9}$ 个项链。因此,根据前一个问题中给出的计数轨道的公式,我们可以设计的项链数量为
\[
\frac{1}{34}(2^{17} + 16 \cdot 2 + 17 \cdot 2^{9}) = \frac{2^{16} + 16}{17}。
\]
注意,由费马小定理,$17$ 能整除 $2^{16} + 2^{4} = (2^{16} - 1) + 17$。
\newpage
\section{Algebra Q5}
\section*{Exercise}
问题 4. 证明立方体的旋转对称群与 $S_4$ 同构。



\subsection*{Solution}
解决方案:设 $G$ 为立方体 $C$ 的旋转对称群,$C$ 嵌入在 $\mathbb{R}^3$ 空间中,且以原点为中心。固定一个面 $f$。注意到 $f$ 可以通过旋转对称映射到 $C$ 的任意六个面。另外,它可以以四种不同的方式完成;注意每个面恰好有四个旋转对称。因此 24。

注意到 $G$ 对主对角线集 $D$ 进行了作用,即连接立方体两个顶点且经过原点的线段。由于 $|D| = 4$,该作用诱导了一个同态 $\phi: G \to S_4$。由于 $|G| = |S_4| = 24$,为了证明 $\phi$ 是一个同构,只需要检查它是单射的。由于没有旋转可以固定四条主对角线,因此该作用是忠实的。因此 $\phi$ 是单射的。因此 $G \cong S_4$。
\newpage
\section{Algebra Q6}
\section*{Exercise}
问题 5. 分类阶为 20 的群。



\subsection*{Solution}
解答:令 $G$ 为阶为 20 的群。如果 $G$ 是阿贝尔群,则根据有限生成阿贝尔群的基本定理,$G$ 必须与 $\mathbb{Z}_2 \times \mathbb{Z}_2 \times \mathbb{Z}_5$ 或 $\mathbb{Z}_4 \times \mathbb{Z}_5$ 同构。

令 $G$ 为非阿贝尔群,阶为 20。令 $n_5$ 为 $G$ 的 5-西洛群的数量。根据西洛定理,$n_5 = 1$。由于存在唯一一个阶为 5 的子群 $N$,$N$ 必定是正常的。令 $H$ 为 2-西洛群。则 $G$ 是 $N$ 和 $H$ 的半直接积,即 $G \cong N \rtimes_{\phi} H$,其中 $\phi: H \to \text{Aut}(N)$。

首先假设 $H$ 与循环群 $\mathbb{Z}_4$ 同构。如果 $\phi$ 是非平凡的同态,则它将 $\mathbb{Z}_4$ 的生成元映射到 $\mathbb{Z}_4$ 的唯一阶为 2 的元素或两个阶为 4 的元素之一。将生成元映射到每个阶为 4 的元素得到同构的非阿贝尔群。令 $\phi_1$ 和 $\phi_2$ 分别为当 $\mathbb{Z}_4$ 的生成元映射到阶为 2 的元素或阶为 4 的元素时得到的同态。由于 $|\text{ker } \phi_1| = 2$ 且 $|\text{ker } \phi_2| = 1$,群 $\mathbb{Z}_5 \rtimes_{\phi_1} \mathbb{Z}_4$ 和 $\mathbb{Z}_5 \rtimes_{\phi_2} \mathbb{Z}_4$ 不同构。

现在假设 $H$ 与克莱因群 $\mathbb{Z}_2 \times \mathbb{Z}_2$ 同构。存在三个非平凡的同态 $\phi: \mathbb{Z}_2 \times \mathbb{Z}_2 \to \text{Aut}(\mathbb{Z}_5)$。它们由将 $\mathbb{Z}_2 \times \mathbb{Z}_2$ 的两个非零元素发送到 $\mathbb{Z}_4$ 中的唯一阶为 2 的元素,并将其他两个元素发送到单位元给出。由克莱因群的对称性,诱导的半直接积是同构的非阿贝尔群。然后 $\mathbb{Z}_5 \rtimes_{\phi} (\mathbb{Z}_2 \times \mathbb{Z}_2)$,其中 $\phi$ 是上面提到的两个同态中的任意一个。该段落中找到的群与前一段落中找到的群不同构,因为 $\mathbb{Z}_5 \rtimes_{\phi_1} \mathbb{Z}_4$ 和 $\mathbb{Z}_5 \rtimes_{\phi_2} \mathbb{Z}_4$ 没有阶为 4 的元素。阶为 4 的元素的缺失迫使 $\mathbb{Z}_5 \rtimes_{\phi} (\mathbb{Z}_2 \times \mathbb{Z}_2)$ 与 $D_{20}$ 同构。
\newpage
\section{Algebra Q7}
\section*{Exercise}
问题6. (I.52) (a) 证明在阿贝尔群的范畴中存在推出(即纤维积)。在这种情况下,两个同态 $f: Y \to X$ 和 $g: Z \to X$ 的纤维积记为 $X \oplus Y$。证明它是因子群 $X \oplus Y = (X \oplus Y) / W$,其中 $W$ 是由所有元素 $(f(z), -g(z))$ 组成的子群,其中 $z \in Z$。 (b) 证明单射同态的推出是单射。



\subsection*{Solution}
解决方案:设 $f: Z \to X$ 和 $g: Z \to Y$ 为群同态。由于 $X$ 和 $Y$ 是阿贝尔群,因此 $X \oplus Y$ 也是阿贝尔群。定义 $h: Z \to X \oplus Y$ 为 $h(z) = (f(z), -g(z))$。$W = \text{Im}(h)$ 是 $X \oplus Y$ 的子群。我们定义 $X \oplus_Z Y = (X \oplus Y) / W$。设 $\pi_1: X \oplus Y \to X \oplus_Z Y$ 和 $\pi_2: X \oplus Y \to X \oplus_Z Y$ 为标准投影。我们需要找到同态 $\phi_1: X \to X \oplus_Z Y$ 和 $\phi_2: Y \to X \oplus_Z Y$,使得 $\phi_1 \circ f = \phi_2 \circ g$。定义 $\phi_1(x) = (x, 0) + W$ 和 $\phi_2(y) = (0, y) + W$。然后 $\phi_1 \circ f(z) = (f(z), 0) + W$ 和 $\phi_2 \circ g(z) = (0, g(z)) + W$。我们需要证明 $(f(z), 0) + W = (0, g(z)) + W$,这意味着 $(f(z), -g(z)) \in W$。这是 $W$ 的定义。因此,图表是可交换的。现在我们需要证明,对于任何阿贝尔群 $A$ 和同态 $\alpha_1: X \to A$ 和 $\alpha_2: Y \to A$,使得 $\alpha_1 \circ f = \alpha_2 \circ g$,存在唯一的同态 $\psi: X \oplus_Z Y \to A$,使得 $\psi \circ \phi_1 = \alpha_1$ 和 $\psi \circ \phi_2 = \alpha_2$。定义 $\psi((x, y) + W) = \alpha_1(x) + \alpha_2(y)$。首先,我们需要证明 $\psi$ 是良定义的。如果 $(x, y) + W = (x', y') + W$,则 $(x-x', y-y') \in W$。因此 $(x-x', y-y') = (f(z), -g(z))$,其中 $z \in Z$。因此 $x-x' = f(z)$ 且 $y-y' = -g(z)$。然后 $\alpha_1(x) + \alpha_2(y) - (\alpha_1(x') + \alpha_2(y')) = \alpha_1(x-x') + \alpha_2(y-y') = \alpha_1(f(z)) + \alpha_2(-g(z)) = \alpha_1(f(z)) - \alpha_2(g(z))$。由于 $\alpha_1 \circ f = \alpha_2 \circ g$,我们有 $\alpha_1(f(z)) - \alpha_2(g(z)) = 0$。因此,$\psi$ 是良定义的。很容易检查出 $\psi$ 是同态。另外,$\psi \circ \phi_1(x) = \psi((x, 0) + W) = \alpha_1(x) + \alpha_2(0) = \alpha_1(x)$ 和 $\psi \circ \phi_2(y) = \psi((0, y) + W) = \alpha_1(0) + \alpha_2(y) = \alpha_2(y)$。对于唯一性,假设存在另一个同态 $\psi'$ 满足条件。然后 $\psi'((x, y) + W) = \psi'((x, 0) + W + (0, y) + W) = \psi'((x, 0) + W) + \psi'((0, y) + W) = \psi'(\phi_1(x)) + \psi'(\phi_2(y)) = \alpha_1(x) + \alpha_2(y) = \psi((x, y) + W)$。因此,$\psi$ 是唯一的。因此,阿贝尔群范畴中存在推出。 (b) 证明推出一个单射同态是单射。假设 $f: Z \to X$ 是单射。我们需要证明 $\phi_2: Y \to X \oplus_Z Y$ 是单射。假设 $\phi_2(y) = 0$,其中 $y \in Y$。则 $(0, y) + W = W$,这意味着 $(0, y) \in W$。因此 $(0, y) = (f(z), -g(z))$,其中 $z \in Z$。这意味着 $f(z) = 0$ 且 $y = -g(z)$。由于 $f$ 是单射,$f(z) = 0$ 意味着 $z = 0$。然后 $y = -g(0) = 0$。因此,$\phi_2$ 是单射。
\newpage
\section{Algebra Q8}
\section*{Exercise}
问题 7. (Lang III.16) 证明一个简单群系统的逆极限,其中同态是满射的,或者是平凡群,或者是简单群。



\subsection*{Solution}
解:设 $(G_i, f_i)$ 为一个简单群系统,其中对于每对 $j > i$,同态 $f_i^j$ 是满射。由于每个 $G_i$ 都是简单的,每个 $f_i^j$ 要么是平凡的,要么是同构。假设 $G_i$ 和 $G_j$ 都是非平凡的。取 $k$ 使得 $k \ge i$ 且 $k \ge j$。由于 $f_i^k$ 和 $f_j^k$ 都是同构,因此 $G_i \cong G_k \cong G_j$。因此,系统中的所有非平凡群都是同构的。设 $(G, f_i)$ 为 $(G_i, f_i)$ 的逆极限。由于 $G$ 是 $\prod G_j$ 的子群,而 $f_i: G \to G_i$ 是投影 $\pi_i: \prod G_j \to G_i$ 到 $G$ 的限制,如果所有 $i$ 都有 $G_i$ 平凡,则 $G$ 也平凡。因此,假设存在 $i$ 使得 $G_i$ 非平凡。在这种情况下,我们将证明 $G \cong G_i$。没有损失一般性地假设所有 $j$ 都有 $G_j$ 非平凡(即 $G_j \cong G_i$ 对所有 $j$)。我们证明 $f_i$ 是同构。
首先,让我们检查 $f_i$ 是否是满射。取 $g_i \in G_i$。对于任何索引 $j$,存在 $k$ 使得 $k \ge j$ 且 $k \ge i$。然后我们取 $g_j = f_j^k(g_k)$,其中 $g_k$ 是 $G_k$ 中唯一的元素,使得 $f_i^k(g_k) = g_i$。如果 $k'$ 也满足 $k' \ge j$ 且 $k' \ge i$,则取 $m$ 使得 $m \ge k$ 且 $m \ge k'$。由于 $f_i^k(g_k) = g_i = f_i^{k'}(g_{k'})$,$g_k$ 和 $g_{k'}$ 必须提升到 $G_m$ 中相同的元素 $g_m$。因此,$f_j^k(g_k) = f_j^m(g_m) = f_j^{k'}(g_{k'})$,这意味着 $g_j$ 不依赖于所选的 $k$。因此,对于 $p \ge q$,$f_q^p(g_p) = g_q$。因此,$(g_j)$ 实际上是 $G$ 中的元素,满足 $f_i((g_j)) = g_i$。因此,$f_i$ 是满射。
现在我们证明 $f_i$ 是单射。假设 $(g_j)$ 在 $f_i$ 的核中。则 $g_i = 1$。对于任何 $j$,存在 $k$ 使得 $k \ge i$ 且 $k \ge j$。由于 $f_i^k(g_k) = g_i = 1$,$g_k = 1$。因此,$g_j = f_j^k(g_k) = 1$。然后 $(g_j)$ 是 $G$ 的单位元,这证明了 $f_i$ 是单射。因此,$G \cong G_i$ 是简单的。
\newpage
\section{Algebra Q9}
\section*{Exercise}
问题 8. (Lang IV.5) 分析以下情况下的不可约性:
(a) 多项式 $x^6 + x^3 + 1$ 在有理数域上是否不可约。
(b) 多项式 $x^2 + y^2 - 1$ 在复数域上是否不可约。
(c) 多项式 $x^4 + 2011x^2 + 2012x + 2013$ 在有理数域上是否不可约。



\subsection*{Solution}
解决方案:

(a) 设 $p(x) = x^6 + x^3 + 1$。注意,多项式 $p(x)$ 可约当且仅当 $q(x) = p(x+1)$ 可约。多项式 $q(x) = (x+1)^6 + (x+1)^3 + 1 = x^6 + 6x^5 + 15x^4 + 21x^3 + 18x^2 + 9x + 3$ 的所有系数(除 $x^6$ 外)都在素数 $3$ 中。由于 $q(x)$ 的常数系数为 $3$,且理想 $(3)$ 是 $q(x)$ 的 Eisenstein 准则,$q(x)$ 在 $\mathbb{Z}$ 上不可约。根据高斯引理,$q(x)$ 在 $\mathbb{Q}$ 上也不可约。因此,多项式 $x^6 + x^3 + 1$ 在有理数上不可约。

(b) 考虑多项式 $p(x,y) = x^2 + y^2 - 1$ 在 $\mathbb{C}$ 上。$p(x,y)$ 是一个变量 $y$ 的多项式,其系数在 $\mathbb{C}[x]$ 中。多项式 $p(x,y)$ 是单项式,其非首项系数在 $\mathbb{C}[x]$ 的素理想 $(x-1)$ 中。由于常数系数 $x^2-1$ 不是 $(x-1)^2$ 的元素,根据 Eisenstein 准则,$q(x,y)$ 在 $\mathbb{C}[x][y]$ 中作为变量 $y$ 的多项式不可约,其系数在 $\mathbb{C}[x]$ 中。因此,$p(x,y)$ 作为两个变量的多项式不可约。

(c) 设 $p(x) = x^4 + 2011x^2 + 2012x + 2013$。根据高斯引理,只需检查 $p(x)$ 在 $\mathbb{Z}$ 上不可约即可。另外,注意到如果 $r(x)$ 在 $\mathbb{Z}$ 上可约,则 $r(x)$ 在 $\mathbb{Z}_2$ 上也可约,其中 $\mathbb{Z}_2[x]$ 是将 $r(x)$ 的系数模 $2$ 减少后的结果。由于 $r(x) = x^4 + x^2 + 1$。由于 $r(x)$ 在 $\mathbb{Z}_2$ 中没有根,因此如果它在 $\mathbb{Z}_2[x]$ 中分解,它将是两个不可约多项式的乘积,每个多项式的次数为 $2$。但是,在 $\mathbb{Z}_2[x]$ 中,只有一个不可约多项式的次数为 $2$,即 $x^2 + x + 1$。由于 $(x^2 + x + 1)^2 = x^4 + x^2 + 1 \neq r(x)$,$r(x)$ 必定在 $\mathbb{Z}_2$ 上不可约。因此,$r(x)$ 在有理数上不可约。
\newpage
\section{Algebra Q10}
\section*{Exercise}
问题 9. (Lang II.6) 设 A 为一个有因式分解的环,p 为一个素元。证明局部环 A(p) 是主理想环。



\subsection*{Solution}
解:设 I 为 A(p) 的一个合适理想,且 $M = \{ \frac{a}{b} : a \in I \text{ 和 } b \notin (p) \}$ 为 A 中的理想。由于 A 是一个有 1 的交换环,任何理想都包含在一个极大理想中;特别地,$I \subset M$。由于 $M = (1)$,I 中的任何元素都可以写成 $\frac{m}{k}$ 的形式,其中 $m \in (p)$ 且 $k \notin (p)$(注意 $\frac{m}{k}$ 是一个单位)。设 $n_0$ 为最小的正整数,使得 $\frac{p^{n_0}}{k} \in I$。我们将证明 $I = \frac{p^{n_0}}{k} A(p)$。由于 I 是一个理想,$\frac{p^{n_0}}{k} \in I$。为了证明逆包含,取 $\frac{mp^k}{b^l} \in I$。由 $n_0$ 的最小性,$k \ge n_0$,因此
\[
\frac{mp^k}{b^l} = \frac{mp^{k-n_0}}{b^l} p^{n_0} \in \left( \frac{p^{n_0}}{1} \right)。
\]
因此,A(p) 的每个理想都是主理想。
\newpage
\section{Algebra Q11}
\section*{Exercise}
设 $F$ 为一个域。证明 $F[[x]]$ 是一个唯一分解整环。



\subsection*{Solution}
设 $R = F[[x]]$。我们将证明 $R$ 是一个主理想环,这实际上是一个更强的陈述。设 $a = \sum a_n x^n$ 为 $R$ 的一个元素。存在 $b = \sum b_n x^n$ 使得 $ab = ba = 1$ 当且仅当 $a_0 \neq 0$。为了证明这一点,我们取 $b_0 = a_0^{-1}$,一旦我们在 $F$ 中选择了 $b_0, \dots, b_{n-1}$,我们就取 $b_n \in F$,使得 $a_0 b_n + \dots + a_n b_0 = 0$。因此,$a_0 \neq 0$ 意味着 $a$ 是一个单位。因此,$M = (x)$ 是 $R$ 的唯一极大理想。由于 $R$ 是一个交换环,且有单位元 $1$,每个理想都必须包含在一个极大理想中。这意味着 $R$ 的每个非零理想都可以表示为 $(x^i)$ 的形式,其中 $i \geq 0$。因此,$R$ 是一个主环(PID),因此也是一个因数分解环(UFD)。
\newpage
\section{Algebra Q12}
\section*{Exercise}
问题 11. 设 $F$ 为一个域。证明 Laurent 多项式环是主理想环。



\subsection*{Solution}
设 $R = F[x,1/x]$ 为 Laurent 多项式环。对于 $f(x) = \sum_{i=-k}^{n}a_ix^i \in R$,其中 $a_{-k} \neq 0$,我们定义 $\text{indeg}(f)$ 为 $k$,如果 $k > 0$,则为零。现在假设 $I$ 是 $R$ 的一个理想。考虑由集合 $S = \{x^{\text{indeg}(r)}r(x) : r(x) \in I\}$ 生成的理想 $\bar{I}$。由于 $\bar{I}$ 是 $F[x]$ 的一个理想,而 $F[x]$ 是一个主理想环 (PID),因此 $\bar{I} = (g(x))$。我们证明 $I = (g(x))$,其中 $(g(x))$ 被认为是 $R$ 的一个理想。取 $I$ 中任意元素 $a(x) \in I$。则有 $x^{\text{indeg}(a)}a(x) \in \bar{I}$,因此存在 $b(x) \in R$,使得 $a(x) = x^{-\text{indeg}(a)}b(x)g(x) \in (g(x))$。另一方面,$S$ 的每个元素都属于 $I$;这是因为 $I$ 是一个理想。因此 $(g(x)) = (S) \subset I$。因此,$I = (g(x))$ 是主理想,这意味着 $R$ 是一个主理想环。
\newpage
\section{Algebra Q13}
\section*{Exercise}
问题 12. (Lang III.17) 设 $n$ 为正整数,$p$ 为素数。证明阿贝尔群 $A_n \approx \mathbb{Z}/p^n\mathbb{Z}$ 在规范同态 $f_{n,m}: A_m \to A_n$ 下形成逆系统,其中 $n \le m$,具体来说,$f_{n,m}(x) = x \pmod{p^n}$。设 $A = \varprojlim A_n$ 为逆极限。证明 $\mathbb{Z}_p$ 映射到每个 $\mathbb{Z}/p^n\mathbb{Z}$ 上是满射的,$\mathbb{Z}_p$ 没有零除数,并且有一个由 $p$ 生成的唯一的极大理想。证明 $\mathbb{Z}_p$ 是有素数分解的,且只有一个素数,即 $p$ 本身。



\subsection*{Solution}
解:自然数集是有序索引系统的一个特例。如果 $n \ge m$,则 $p^n\mathbb{Z} \subseteq p^m\mathbb{Z}$,因此 $q_n: \mathbb{Z}/p^n\mathbb{Z} \to \mathbb{Z}/p^m\mathbb{Z}$ 由 $a + p^n\mathbb{Z} \mapsto a + p^m\mathbb{Z}$ 给出,其中 $a \in \mathbb{Z}$ 是一个良定义的满射同态。另外,如果 $n \ge m \ge k$,
\[q_m^n(q_k^m(a + p^k\mathbb{Z})) = q_m^n(a + p^m\mathbb{Z}) = a + p^n\mathbb{Z} = q_k^n(a + p^k\mathbb{Z}).\]
因此 $(A_n, q_m^n)$ 是一个逆系统。
用 $f_{ij}$ 表示从 $\mathbb{Z}_p$ 到 $A_j$ 的同态。固定索引 $i$ 并取 $a_i \in A_i$。递归定义 $a_{i+j}$,使得 $q_{i+j}^{i+j+1}(a_{i+j+1}) = a_{i+j}$,从 $j=1$ 开始。另外,对于 $i \ge j$,定义 $q_j(a_j)$。由此可知,对于任何 $i \ge j$,$q_j(a_i) = a_j$,这意味着 $(a_j) \in \varprojlim \mathbb{Z}/p^j\mathbb{Z}$。由于 $f_j((a_i)) = a_j$,$f_j$ 是满射。
为了检查 $A_p$ 是否不包含任何零除数,取 $(a_i + p^i\mathbb{Z})$ 和 $(b_i + p^i\mathbb{Z})$ 在 $\mathbb{Z}_p$ 中,其乘积为零。假设存在 $r,s$,使得 $p^r$ 不整除 $a_r$ 且 $p^s$ 不整除 $b_s$。因此,$p$ 不整除 $a_r$ 且 $p$ 不整除 $b_s$。然后 $p^{r+s}$ 不整除 $a_r b_s$,这意味着 $a_r b_s + p^{r+s}\mathbb{Z}$ 非零。但是,这与乘积 $(a_i + p^i\mathbb{Z})$ 和 $(b_i + p^i\mathbb{Z})$ 为零相矛盾。因此,$(a_i + p^i\mathbb{Z})$ 或 $(b_i + p^i\mathbb{Z})$ 必须为零。因此,$\mathbb{Z}_p$ 没有零除数。
为了证明理想 $M$ 由 $p$ 生成是 $\mathbb{Z}_p$ 的唯一最大理想,取 $\mathbb{Z}_p$ 中不在 $M$ 中的元素 $(a_i + p^i\mathbb{Z})$。然后 $p$ 不整除 $a_i$,这意味着 $(a_i, p^i) = 1$。对于每个 $i$,取 $b_i$,使得 $a_i b_i = 1 \pmod{p^i}$。由于 $p^i$ 整除 $a_i b_i - 1$ 和 $p^i$,且 $p^i$ 整除 $p^i$,因此 $p^i$ 整除 $a_i b_i - 1$。因此,$p^i$ 整除 $b_i - b_j$,这意味着 $q_j^i(b_i + p^i\mathbb{Z}) = b_j + p^j\mathbb{Z}$。因此,$(b_i + p^i\mathbb{Z})$ 是 $(a_i + p^i\mathbb{Z})$ 在 $\mathbb{Z}_p$ 中的逆元。由于 $M$ 之外的任何元素都是单位,$M$ 是 $\mathbb{Z}_p$ 的唯一最大理想。因此,$\mathbb{Z}_p$ 是一个局部环。
由于 $\mathbb{Z}_p$ 是一个交换环,任何理想都包含在一个最大理想中。另一方面,$M$ 是 $\mathbb{Z}_p$ 的唯一最大理想,因此任何理想都包含在 $M$ 中。这意味着 $\mathbb{Z}_p$ 的每个理想都是主理想。因此,$\mathbb{Z}_p$ 是一个主理想环 (PID)。特别地,$\mathbb{Z}_p$ 是一个因式分解环 (UFD)。由于 $M$ 是一个素理想,$p$ 是 $\mathbb{Z}_p$ 中的素数。假设 $q$ 是一个素数。那么理想 $(q)$ 包含在 $M$ 中。因此,$q = up^k$,其中 $u$ 是一个单位。由于 $q$ 是不可约的,$k=1$,这意味着 $q$ 与 $p$ 关联。因此,
\newpage
\section{Algebra Q14}
\section*{Exercise}
问题 13. 设 $\omega$ 为 $x^2 - x + 1$ 的一个根。证明 $\mathbb{Z}[\omega]$ 是欧几里得整环。



\subsection*{Solution}
解:设 $R = \mathbb{Z}[\omega]$。由于 $x^2 - 1 = (x - 1)(x^2 + x + 1)$,因此 $\omega$ 是一个六次方根,事实上是一个原始的六次方根。我们可以假设没有失去普遍性,$\omega$ 是主要的六次方根。因此,$\mathbb{Z}$ 由以下线的交点组成:

(i) 与实轴平行的线,交于虚轴于 $ib\sqrt{3}$,其中 $b \in \mathbb{Z}$;
(ii) 斜率为 $\sqrt{3}$ 的线,交于虚轴于 $ib\sqrt{3}$,其中 $b \in \mathbb{Z}$;
(iii) 斜率为 $-\sqrt{3}$ 的线,交于虚轴于 $ib\sqrt{3}$,其中 $b \in \mathbb{Z}$。

因此,$R$ 的点在 $\mathbb{C}$ 中形成一个由等边单极三角形组成的网格。这意味着对于任何 $x \in \mathbb{C}$,都存在 $q \in R$ 使得 $|x - q| \le 1$。因此,对于 $a, b \in R$ 且 $b \neq 0$,存在 $q \in R$ 使得 $|a/b - q| \le \frac{1}{2}$。设 $r = a - qb$,则有
\[ |r| = |a - qb| = |a/b - q||b| \le \frac{\sqrt{3}}{2}|b| < |b|. \]
因此 $a = qb + r$,其中 $|r| < |b|$。因此,$R$ 是一个欧几里得域。
\newpage
\section{Algebra Q15}
\section*{Exercise}
问题 14. 设 $R$ 为半单环,$L \subset R$ 为左理想。证明存在幂等元 $e$ 使得 $L = Re$。



\subsection*{Solution}
解:将$L$视为$R$的左$R$子模。由于$R$作为自身的模块是半单的,所以存在$R$的左$R$子模$L'$,使得$R = L \oplus L'$。取$L$中的$e$和$L'$中的$e'$,使得$1 = e + e'$。那么我们有
\[e + 0 = e = e(e + e') = e^2 + ee'.\]
\newpage
\section{Algebra Q16}
\section*{Exercise}
问题 15. 确定所有半单环的同构类,且这些环的阶为 1008。其中有多少个是可交换的?



\subsection*{Solution}
解答:设 $R$ 为半单环,且 $|R|=1008=2^4\cdot3^2\cdot7$。由于 $R$ 为有限环,因此它是阿廷环。因此,根据阿廷-韦德伯恩定理,$R$ 是有限多个 $n_i\times n_i$ 矩阵环的乘积,其中这些矩阵的元素来自除环 $R_i$。由于 $R$ 为有限环,因此对于每个索引 $i$,$R_i$ 也为有限环。因此,每个 $R_i$ 都是域。特征为 2 的域上的矩阵环的可能乘积为 $M_2(\mathbb{F}_2)$、$\mathbb{F}_{16}$、$\mathbb{F}_4 \times \mathbb{F}_4$、$\mathbb{F}_2 \times \mathbb{F}_2 \times \mathbb{F}_2 \times \mathbb{F}_2$ 和 $\mathbb{F}_2 \times \mathbb{F}_8 \times \mathbb{F}_2$。使用特征为 3 的域代替特征为 2 的域,矩阵环的可能乘积为 $\mathbb{F}_9$ 和 $\mathbb{F}_3 \times \mathbb{F}_3$。在特征为 7 的情况下,只有一个这样的乘积,即 $\mathbb{F}_7$。将我们之前获得的乘积组合起来,我们可以得到 1008 阶半单环的每个同构类的代表。共有 10 个同构类。只有包含一个 $n \times n$ 矩阵环作为因子且 $n > 1$ 的代表不是可交换的。因此,同构的半单环中有 8 个是可交换的。
\newpage
\section{Algebra Q17}
\section*{Exercise}
问题16. 设$P$为任意环$R$上的一个循环投射模块。证明存在$R$的某个幂等元$e$使得$P \approx Re$。



\subsection*{Solution}
解决方案:很容易检查 $g_x$ 是一个 $R$-模同态。还要注意 $\ker(g_x) = \text{Ann}(x)$。因此,我们有以下短正合序列:
\[ 0 \to \text{Ann}(x) \to R \xrightarrow{g_x} Rx \to 0. \]
由于 $Rx$ 是投射的,所以存在一个同态 $f: Rx \to R$,使得 $g_x \circ f = 1_{Rx}$。这意味着 $f(x)x = x$。定义 $e = f(x)$ 为 $f(x)$,并观察到 $(e-1)x = 0$。所以 $e-1 = \text{Ann}(x)$。对于 $e \in \text{Ann}(x)$,使得 $e = e+e$,我们有:
\[ e = (e+a)e = e^2 + ae = e^2 + af(x) = e^2 + f(ax) = e^2 + f(0) = e^2. \]
因此 $e$ 是一个幂等元。由于 $g_x \circ f = 1_{Rx}$,映射 $f$ 是单射的。因此 $Re \approx Rx$,其中 $Re$ 是 $Rx$ 由 $f$ 映射的像。
\newpage
\section{Algebra Q18}
\section*{Exercise}
问题 17. (Lang III.10) (a) 设 $A$ 为一个有单位元的交换环。如果 $\mathfrak{p}$ 是一个素理想,且 $S = A - \mathfrak{p}$ 是环 $A$ 中 $\mathfrak{p}$ 的补集,则 $S^{-1}M$ 表示为 $M_{\mathfrak{p}}$。证明模块 $M$ 到所有局部化 $M_{\mathfrak{m}}$ 的直接积的自然映射
\[ M \to \prod_{\mathfrak{m}} M_{\mathfrak{m}} \]
是单射的,其中 $\mathfrak{m}$ 遍历所有的极大理想。


\subsection*{Solution (by LLM)}
由LLM生成的解答可能不准确,请自行验证。

\textbf{解答:}

我们需要证明自然映射
\[
M \to \prod_{\mathfrak{m}} M_{\mathfrak{m}}
\]
是单射,其中$\mathfrak{m}$遍历所有极大理想。

设$M$是一个$A$-模,$\mathfrak{m}$是$A$的极大理想。对于每个$\mathfrak{m}$,我们有局部化$M_{\mathfrak{m}} = S^{-1}M$,其中$S = A - \mathfrak{m}$。自然映射$M \to M_{\mathfrak{m}}$由$m \mapsto \frac{m}{1}$给出。

考虑自然映射
\[
\phi: M \to \prod_{\mathfrak{m}} M_{\mathfrak{m}}, \quad \phi(m) = \left( \frac{m}{1} \right)_{\mathfrak{m}}.
\]
我们需要证明$\phi$是单射,即如果$\phi(m) = 0$,则$m = 0$。

假设$\phi(m) = 0$,即对于所有极大理想$\mathfrak{m}$,$\frac{m}{1} = 0$在$M_{\mathfrak{m}}$中。这意味着对于每个$\mathfrak{m}$,存在$s_{\mathfrak{m}} \in S = A - \mathfrak{m}$,使得$s_{\mathfrak{m}} m = 0$。

考虑理想$I = \{ a \in A \mid a m = 0 \}$。由于$s_{\mathfrak{m}} \in I$且$s_{\mathfrak{m}}  otin \mathfrak{m}$,$I$不在任何极大理想$\mathfrak{m}$中。因此,$I$必须是整个环$A$,即$1 \in I$。这意味着$1 \cdot m = 0$,即$m = 0$。

因此,$\phi$是单射。
\newpage
\section{Algebra Q19}
\section*{Exercise}
问题18. 找出满足以下条件的模块M。
1. M既是投射模块,也是注射模块。
2. M是投射模块,但M不是注射模块。
3. M是注射模块,但M不是投射模块。
4. M既不是投射模块,也不是注射模块。



\subsection*{Solution}
解决方案:
\textbf{1. }设 $F$ 为一个域。令 $M$ 为 $F$ 模块 $F$。由于 $M$ 是一个自由的 $F$ 模块,因此它是投射的。另一方面,$M$ 是可除的,因此由于 $F$ 是一个 PID,$M$ 是可注射的。
\textbf{2. }令 $M$ 为 $\mathbb{Z}$ 模块 $\mathbb{Z}$。由于 $M$ 是自由的,因此它是投射的。另一方面,$M$ 不是可注射的:这是因为 $\mathbb{Z}$ 是一个 PID 且 $M$ 不是 $\mathbb{Z}$ 模块的可除模块。
\textbf{3. }令 $M$ 为 $\mathbb{Z}$ 模块 $\mathbb{Q}$。由于 $\mathbb{Q}$ 是一个可除的阿贝尔群,因此它是一个可注射的 $\mathbb{Z}$ 模块。让我们证明 $\mathbb{Q}$ 不是投射的。假设,通过矛盾,$\mathbb{Q}$ 是投射的。那么它是自由 $\mathbb{Z}$ 模块的直接和。令 $B$ 为 $F$ 的基数,令 $i: \mathbb{Q} \to F$ 为包含映射。存在 $n \in \mathbb{N}$,$b_1, \dots, b_n \in B$,和 $z_1, \dots, z_n \in \mathbb{Z}$,使得 $i(1) = \sum_{i=1}^n z_i b_i$。由于 $i(1) = m i(1)/m$ 对于每个 $m \in \mathbb{N}$ 和 $i(1) = \sum_{i=1}^n z_i b_i$,我们推断出 $m$ 必须除以 $z_i$ 对于每个 $i = 1, \dots, n$。因此,对于每个 $i = 1, \dots, n$,$z_i = 0$,因此 $i(1) = 0$。现在,如果 $f: F \to \mathbb{Q}$ 是投射映射,我们有 $1 = f(i(1)) = f(0) = 0$,这是一个矛盾。因此,$\mathbb{Q}$ 不是投射的。
\textbf{4. }令 $M$ 为 $\mathbb{Z}/2\mathbb{Z}$ 模块 $\mathbb{Z}/2\mathbb{Z}$。考虑 $\mathbb{Z}/4\mathbb{Z}$ 模块的序列
\[ 0 \to \mathbb{Z}/2\mathbb{Z} \xrightarrow{\text{mult by } 2} \mathbb{Z}/4\mathbb{Z} \xrightarrow{\text{mod } 2} \mathbb{Z}/2\mathbb{Z} \to 0 \]
很容易检查上述序列实际上是一个短正合序列。然而,它不分裂,因为 $\mathbb{Z}/4\mathbb{Z}$ 不同构于 $\mathbb{Z}/2\mathbb{Z} \times \mathbb{Z}/2\mathbb{Z}$。因此,$M$ 既不是投射的,也不是可注射的。
\newpage
\section{Algebra Q20}
\section*{Exercise}
问题19. 证明局部环上的有限生成的投射模块是自由模块。



\subsection*{Solution}
解:设 $R$ 为局部环,$M$ 为有限生成的射影 $R$- 模块。取最小的 $n$,使得 $M = Rm_1 + \dots + Rm_n$,其中 $m_i \in M$。由于 $R^n$ 是自由的,因此存在 $R$- 模同态 $\phi : R^n \to M$,使得以下短正合序列成立:
\[ 0 \to K \xrightarrow{} R^n \xrightarrow{\phi} M \to 0, \]
其中 $K$ 是 $\phi$ 的核。由于 $M$ 是射影的,因此上述序列分裂,因此 $M \oplus K \cong R^n$。设 $\mathfrak{m}$ 为 $R$ 的最大理想。将 $M \oplus K \cong R^n$ 张量化为 $R/\mathfrak{m}$,我们发现 $(R/\mathfrak{m})^n \cong M/\mathfrak{m} \oplus K/\mathfrak{m}$ 是 $R/\mathfrak{m}$ 上的向量空间。元素 $m_1, \dots, m_n$ 生成 $M/\mathfrak{m}M$,其中 $\overline{m_i} = m_i + \mathfrak{m}M$。假设现在 $r_i \in R$ 且 $r_i \in R/\mathfrak{m}$,这意味着 $\sum r_i m_i \in \mathfrak{m}M$,因此 $r_i \in M$ 对于所有 $i$。因此,元素 $m_1, \dots, m_n$ 在 $M/\mathfrak{m}M$ 中是线性无关的,因此是基。因此 $\dim M/\mathfrak{m}M = n$,这意味着 $N/\mathfrak{m}N$ 是平凡的。由于 $N = \mathfrak{m}N$ 且 $R$ 是具有最大理想 $\mathfrak{m}$ 的局部环,中山引理意味着 $N = 0$。因此,$M \cong R^n$ 是自由的 $R$- 模块。
\newpage
\section{Algebra Q21}
\section*{Exercise}
问题20. (Lang III.19) 设$(A_i, f_{ij})$为一个有向模块族。设$a_k \in A_k$,且假设$a_k$在直极限$A$中的像为$0$。证明存在某个$m \ge k$,使得$f_{km}(a_k) = 0$。换句话说,对于某个群$A$中的任意一个元素,在直极限中消失的现象已经可以在原始数据中观察到。



\subsection*{Solution}
解决方案:对于索引 $i$,令 $f_i: A_i \to A$ 为直接极限给出的映射。令 $S = \{a_i\}$,对于 $x_i \in A_i$,用 $x_i'$ 表示直接和中第 $i$ 个分量为 $x_i$,其余分量为零的元素。令 $N$ 为 $S$ 的子群,由元素 $(...,0,x_i,-f_{ij}(x_i),0,...)$ 生成,其中 $x_i \in A_i$,$-f_{ij}(x_i) \in A_j$,且 $i \le j$。事实 $f_k(a_k) = 0$ 意味着 $a_k \in N$。因此,我们可以将 $a_k$ 写为
\[
(...,0,a_k,...) = \sum (...,0,a_i,-f_{ij}(a_i),0,...)
\]
其中 $r \ge 1$,且 $i_t \le j_t$ 对于 $1 \le t \le r$。尽管上述索引 $i_t$ 和 $j_t$ 可能分布在不同的分量中,但当 $s  e k$ 时,第 $s$ 个分量的和为零,当 $s = k$ 时,第 $s$ 个分量的和为 $a_k$。因此,对于 $m \ge \max\{k,j_1,...,j_r\}$(该值必定存在),我们有
\[
f_{km}(a_k) = f_{km}(a_k) - \sum f_{j_t m}(f_{i_t j_t}(a_{i_t})) + \sum f_{i_t m}(a_{i_t}) - \sum f_{j_t m}(f_{i_t j_t}(a_{i_t}))
\]
\[
= \sum (f_{i_t m}(a_{i_t}) - f_{j_t m}(f_{i_t j_t}(a_{i_t})) + f_{i_t m}(a_{i_t}) - f_{j_t m}(a_{i_t}))
\]
\[
= 0。
\]
因此,$m$ 就是我们要找的索引。
\newpage
\section{Algebra Q22}
\section*{Exercise}
问题 21. (Lang III.24) 证明任何模块都是有限生成子模块的直接极限。



\subsection*{Solution}
解:设 $R$ 为一个环,$M$ 为 $R$-模块。对于 $M$ 的任意有限子集 $S$,记 $M_S$ 为由 $S$ 生成的 $M$ 的有限子模块。$M$ 的有限子集形成一个有向索引系统。对于 $M$ 的有限子集 $S$ 和 $T$,其中 $S \subseteq T$,记 $i_{S,T}$ 为从 $M_S$ 到 $M_T$ 的包含映射。$(M_S, i_{S,T})$ 是有限生成 $R$-模块的有向系统。我们证明 $(M, i_S)$ 是 $(M_S, i_{S,T})$ 的直接极限,其中 $i_S: M_S \to M$ 是包含映射。对于 $S \subset T$,$i_T \circ i_{S,T} = i_S$。设 $(N, f_S)$,其中 $N$ 是 $R$-模块,对于 $M$ 的每个有限子集 $S$,$f_S: M_S \to N$ 是 $R$-模块同态,且 $f_T \circ i_{S,T} = f_S$。定义 $f: M \to N$ 如下。对于 $m \in M$,设 $f(m) = f_{\{m\}}(m)$。如果 $S$ 是包含 $m$ 的 $M$ 的有限子集,则 $f_S(m) = f_{\{m\}}(m)$。因此,如果 $a, b \in M$ 且 $c \in R$,则 $f(a+b) = f_{\{a,b\}}(a+b) = f_{\{a,b\}}(a) + f_{\{a,b\}}(b) = f(a) + f(b)$。因此 $\phi$ 是同态。对于 $M$ 的有限子集 $S$ 和 $m \in M_S$,$f(i_S(m)) = f(m) = f_{\{m\}}(m) = f_S(m)$。因此,$M$ 是其有限生成子模块的直接极限。
\newpage
\section{Algebra Q23}
\section*{Exercise}
问题22. (Lang III.21) 设$(M'_i, f_{ij}')$和$(M_i, g_{ij})$是模块的有向系统。模块$(M'_i)$到$(M_i)$的同态指的是一族同态$u_i : M'_i \to M_i$,其中每个$i$都与$f_{ij}'$和$g_{ij}$交换。假设我们有一个精确序列
\[0 \to M'_i \xrightarrow{\alpha_i} M_i \xrightarrow{\beta_i} M''_i \to 0\]
有向系统的精确序列,意味着对于每个$i$,序列
\[0 \to M'_i \xrightarrow{\alpha_i} M_i \xrightarrow{\beta_i} M''_i \to 0\]
是精确的。证明直接极限保持精确性,即
\[0 \to \varinjlim M'_i \to \varinjlim M_i \to \varinjlim M''_i \to 0\]
是精确的。



\subsection*{Solution}
解:令 $M'$、$M$ 和 $M''$ 分别表示直极限 $\varinjlim M'_i$、$\varinjlim M_i$ 和 $\varinjlim M''_i$。我们知道 $(M'_i, \alpha_i)$ 是一个有向系统的模块,$\alpha_i : M'_i \to M_i$ 与 $g_{ij}'$ 和 $h_{ij}'$ 通勤。只要证明序列 $M' \to M \to M''$ 在 $M$ 处是精确的,因为在 $M'$ 和 $M''$ 处的精确性可以通过考虑序列 $0 \to M' \to M$ 和 $M \to M'' \to 0$ 得到。

我们将证明 $\ker(v) = \text{Im}(u)$。对于每个 $m' \in M'$,存在 $i$ 和 $m'_i \in M'_i$,使得 $m' = f_i(m'_i)$。考虑以下交换图

\[\begin{array}{ccccccccc}
0 & \to & M'_i & \xrightarrow{\alpha_i} & M_i & \xrightarrow{\beta_i} & M''_i & \to & 0 \\
& & \downarrow f_i & & \downarrow g_i & & \downarrow h_i & & \\
0 & \to & M' & \xrightarrow{u} & M & \xrightarrow{v} & M'' & \to & 0
\end{array}\]

我们得到
\[v(u(m')) = v(u(f_i(m'_i))) = v(g_i(\alpha_i(m'_i))) = h_i(\beta_i(\alpha_i(m'_i))) = h_i(0) = 0;\]

因此序列是精确的。因此 $\text{Im}(u) \subseteq \ker(v)$。为了证明反向包含,取 $m \in \ker(v)$。然后取 $i$ 和 $m_i \in M_i$,使得 $m = g_i(m_i)$。由于 $h_i(\beta_i(m_i)) = v(g_i(m_i)) = 0$,存在 $j \ge i$,使得 $h_j(\beta_i(m_i)) = 0$。因此 $g_j(\beta_i(m_i)) = 0$。由于 $g_j(\beta_i(m_i)) \in \ker(v_j) = \text{Im}(u_j)$,存在 $m'_j \in M'_j$,使得 $u_j(m'_j) = g_j(m_i)$。取 $m' = f_j(m'_j) \in M'$,

\[u(m') = u(f_j(m'_j)) = g_j(u_j(m'_j)) = g_j(g_j(m_i)) = g_j(m_i) = m.\]

因此 $\ker(v) \subseteq \text{Im}(u)$,这意味着序列 $M' \to M \to M''$ 在 $M$ 处是精确的。
\newpage
\section{Algebra Q24}
\section*{Exercise}
问题 23. (Lang III.23) 设 (Mi) 为一个环上的有向模块族。对于任意模块 N,证明:

lim
←−Hom(N, Mi) = Hom(N, lim
←−Mi)。



\subsection*{Solution}
解决方案:对于 i ≤ j,我们用 fji:Mj → Mi 表示模块 (Mi) 的有向族中的同态,并用 M 表示其逆极限。对于 i ≤ j,我们用 ¯fji:Hom(N, Mj) → Hom(N, Mi) 表示由 ¯fji(φ)(x) = fji(φ(x)) 对于 φ ∈ Hom(N, Mj) 和 x ∈ N 给出的同态。由于 (Mi) 是模块的有向族,因此 (Hom(N, Mi), ¯fji) 也是模块的有向族。

我们用 ¯fi 表示从 Hom(N, M) 到 Hom(N, Mi) 的同态,由 ¯fi(φ)(x) = fi(φ(x)) 给出。我们只需要证明 (Hom(N, M), ¯fi) 是模块的有向族 (Hom(N, Mi), ¯fji) 的逆极限。

如果 φ ∈ Hom(N, M) 且 x ∈ N,我们有

( ¯fji( ¯fj(φ)))(x) = fji( ¯fj(φ)(x)) = fji(fj(φ(x))) = fi(φ(x)) = ¯fi(φ)(x),

对于任何 i ≤ j。因此 ¯fji ◦ ¯fj = ¯fi 对于 i ≤ j(即上述图表中的上三角形是可交换的)。现在假设 (A, αi),其中 αi:A → Hom(N, Mi) 满足 ¯fji ◦ αj = αi。我们定义 α:A → Hom(N, M) 由 α(a)(x) = (αi(a)(x)) 对于 a ∈ A 和 x ∈ N。若 i ≤ j,fji(αj(a)(x)) = αi(a)(x) 对于每个 a ∈ A 和 x ∈ N。因此 (αi(a)(x)) ∈ M 对于每个 a ∈ A 和 x ∈ N,这意味着 α(a):N → M 是一个良定义的映射。映射 α 在以下图表中用虚线表示:

Hom(N, Mj)
Hom(N, Mi)
Hom(N, M)
A
¯
fji
¯
fj
¯
fi
α
αj
αi

对于 A 中的每个 a,映射 αi(a) 都是同态,因此映射 α(a) 也是同态。因此,我们有 α 是良定义的。事实上 α 是模块同态,来自于 αi 是每个索引 i 的模块同态。最后,对于任何索引 i 和 a ∈ A,我们有

¯fi(α(a))(x) = fi(α(a)(x)) = fi((αj(a)(x))) = αi(a)(x),

对于所有 x ∈ N。因此 ¯fi ◦ α = αi 对于所有索引 i(即 α 保持上述图表的可交换性)。因此 (Hom(N, M), ¯fi) 是模块的有向族 (Hom(N, Mi), ¯fji) 的逆极限。

■
\newpage
\section{Algebra Q25}
\section*{Exercise}
问题 24. (Lang V.13) 如果某个单变量多项式 $f(x) \in k(x)$ 在某个分裂域中的根是不同的,并且形成一个域,则 $\text{char}(k) = p$ 且 $f(x) = x^{p^m} - x$,其中 $m \ge 1$。



\subsection*{Solution}
解:设 $F = \{r_1, \dots, r_n\}$ 为 $f$ 的根集。由于 $F$ 是一个域,对于任何 $k \in \mathbb{N}$,$k \cdot 1$ 都是 $f$ 的根。由于多项式只有有限个根,$\text{char}(k) = p$。由于 $f$ 在 $F$ 上分解,即 $f(x) = (x - r_1) \dots (x - r_n)$,且 $F$ 由 $f$ 的根生成,$F$ 必定是 $f$ 的分解域。设 $\mathbb{F}_p$ 为 $F$ 的素域。由于 $F$ 是 $\mathbb{F}_p$ 上的有限维向量空间,$n = |F| = p^m$。我们还知道,阶为 $p^m$ 的有限域是多项式 $x^{p^m} - x$ 的分解域。因此,$f(x) = x^{p^m} - x$。
\newpage
\section{Algebra Q26}
\section*{Exercise}
问题25. (Lang V.14) 设 $\text{char}(K) = p$。设 $L$ 是 $K$ 的有限扩张,且 $[L : K]$ 与 $p$ 互质。证明 $L$ 在 $K$ 上是可分的。



\subsection*{Solution}
解:取$L$中的一个元素$\alpha$。由于$L : K$是有限的,$\alpha$是代数的。设$f(x) \in K[x]$是$\alpha$在$K$上的不可约多项式。由于$m = \deg f$能整除$[L : K]$,我们有$(m, p) = 1$。设$f'$是$K[x]$中一个可分多项式,使得$f(x) = (f'(x))^{p^e}$,其中$e$是非负整数。这意味着$m = \deg f' \cdot p^e \deg f_{\text{sep}}$,因此$p^e$能整除$m$。事实$(m, p) = 1$迫使$e$为零。因此$f = f_{\text{sep}}$在$K$上是可分的。
\newpage
\section{Algebra Q27}
\section*{Exercise}
问题 26. (Lang V.15) 假设 $\text{char}(K) = p$。令 $a \in K$。如果 $a$ 在 $K$ 中没有 $p$ 次方根,证明对于所有正整数 $n$,$x^{p^n} - a$ 在 $K[x]$ 中是不可约的。



\subsection*{Solution}
解:设 $n$ 为正整数,$f(x) = x^{p^n} - a$。如果 $\beta_1$ 和 $\beta_2$ 是 $f$ 在某个分裂域中的根,则有 $(\beta_1 - \beta_2)^{p^n} = \beta_1^{p^n} - \beta_2^{p^n} = a - a = 0$,因此 $\beta_1 = \beta_2$。因此,$f$ 是完全不可分的,因此在 $K$ 上的某个分裂域 $F$ 中存在 $\beta$,使得 $f(x) = (x - \beta)^{p^n} = x^{p^n} - \beta^{p^n}$。如果 $g$ 是 $\beta$ 在 $K$ 上的不可约多项式,则在 $F$ 中,$g(x) = (x - \beta)^k$,其中 $k$ 为某个正整数。由于 $f$ 是不可约的且完全不可分的,$g$ 的次数必定是 $p$ 的幂。由于 $g$ 能整除 $K[x]$ 中的 $f$,因此有 $k \le p^n$。如果 $k < p^n$,则 $g(x) = (x - \beta)^k$,因此 $g$ 是不可约的。设 $k < n$,由此可知 $\beta^{p^k}$ 是 $g$ 的系数,因此 $\beta^{p^k} \in K$。如果 $a = \beta^{p^n}$,则有 $a = (\beta^{p^k})^{p^{n-k}} \in K$。因此,有 $a = \alpha^{p^k}$,其中 $\alpha \in K$。这与 $a$ 在 $K$ 中没有 $p$ 次方根的事实相矛盾。
\newpage
\section{Algebra Q28}
\section*{Exercise}
问题27. (Lang V.16) 设 $\text{char}(K) = p$。设 $\alpha$ 是 $K$ 的代数数。证明 $\alpha$ 可分割当且仅当对于所有正整数 $n$,$K(\alpha) = K(\alpha^{p^n})$。



\subsection*{Solution}
解决方案:首先,我们假设 $\alpha$ 是可分的。鉴于 $K(\alpha)$ 是可分的,足以证明 $\alpha$ 是可分的。由于 $\alpha$ 是多项式 $x^{p^n} - \alpha^{p^n} \in K(\alpha^{p^n})[x]$ 的根,$\alpha$ 在 $K(\alpha^{p^n})$ 上的不可约多项式 $g(x)$ 能整除 $x^{p^n} - \alpha^{p^n}$。因此,在 $K(\alpha^{p^n})[x]$ 中,$g(x) = (x - \alpha)^{p^n}$。如果 $g(x)$ 是 $\alpha$ 在 $K$ 上的不可约多项式,则 $g(x)$ 能整除 $K[x]$ 中的 $(x - \alpha)^{p^n}$。由于 $\alpha$ 是可分的,$g'(x) \neq 0$。因此,$\alpha \in K(\alpha^{p^n})$。

另一方面,假设对于所有正整数 $n$,$K(\alpha) = K(\alpha^{p^n})$。令 $a(x)$ 为 $\alpha$ 在 $K$ 上的不可约多项式。存在一个可分多项式 $a_{\text{sep}}(x) \in K[x]$ 和一个非负整数 $e$,使得 $a(x) = a_{\text{sep}}(x^{p^e})$。由于 $\alpha$ 是 $a(x)$ 的根,$\alpha^{p^e}$ 是 $a_{\text{sep}}(x)$ 的根。由于域 $K(\alpha)$ 是不可约的,$a_{\text{sep}}(x)$ 也是不可约的。因此,$a_{\text{sep}}(x)$ 是 $\alpha^{p^e}$ 在 $K$ 上的不可约多项式。由此可知
\[
\text{deg } a_{\text{sep}}(x) = [K(\alpha^{p^e}) : K] = [K(\alpha) : K] = \text{deg } a(x)。
\]
这意味着 $k = 0$,因此 $\alpha(x) = a_{\text{sep}}(x)$。因此,$\alpha(x)$ 是可分的。
\newpage
\section{Algebra Q29}
\section*{Exercise}
问题 28. (Lang V.18) 证明有限域中的每个元素都可以表示为该域中两个平方和。



\subsection*{Solution}
设 $F$ 为有限域。若 $\text{char}(F) = 2$,则弗罗贝尼乌斯同态是满射的,因此对于任何 $y \in F$,存在 $x \in F$ 使得 $y = x^2 = x^2 + 0^2$。现在假设 $\text{char}(F) = p$ 为奇素数。则 $|F|$ 为奇数,因此 $|F^\times| = 2k$,其中 $k$ 为某个自然数。此外,$F^\times$ 是循环群;设 $F^\times = \langle a \rangle$。注意到 $a$ 的每个偶数次方都是平方数。由于 $0$ 也是 $F$ 中的平方数,因此至少有 $k + 1$ 个元素是平方数。设 $S$ 为 $F$ 中所有平方数的集合。对于 $F$ 中的任意元素 $y$,以下不等式成立:
\[
|y - S| = |S| = k + 1 \geq \frac{|F|}{2}
\]
根据鸽巢原理,存在 $s \in (y - S) \cap S$,这意味着 $s = s_1^2$,其中 $s_1 \in F$,且 $s = y - s_2^2$,其中 $s_2 \in F$。因此 $y = s_1^2 + s_2^2$。
\newpage
\section{Algebra Q30}
\section*{Exercise}
问题29. (Lang V.24) 证明原始元素定理不一定对有限不可分扩张成立。



\subsection*{Solution}
解:设 $F = \mathbb{Z}_p(Y, Z)$,其中 $Y$ 和 $Z$ 是 $\mathbb{Z}_p$ 上的两个代数独立的超越元素。设 $E$ 为 $F$ 的一个代数闭包。考虑多项式 $p(x) = x^p - Y$ 和 $q(x) = x^p - Z$ 在 $F[x]$ 中。由于 $p(x)$ 和 $q(x)$ 是关于素理想 $(Y)$ 和 $(Z)$ 的 Eisenstein 多项式,因此这两个多项式在 $F[x]$ 中是不可约的。根据高斯引理,它们也是 $F$ 上不可约的。设 $\alpha$ 和 $\beta$ 分别为 $p(x)$ 和 $q(x)$ 在 $E$ 中的根。考虑域扩张 $F(\alpha, \beta)/F$。由于 $p(x) = (x - \alpha)^p$ 在 $F(\alpha)[x]$ 中(在 $F$ 中约化),我们将得到 $F \subset F(\alpha)$,因此 $\beta^p = Z \in F(\alpha)$,其中 $c_i \in F$。但这将意味着 $Z = \sum c_i \alpha^i$,这不可能发生,因为 $Y$ 和 $Z$ 是代数独立的。因此,$q(x)$ 在 $F(\alpha)$ 上是不可约的。这意味着
\[[F(\alpha, \beta) : F] = [F(\alpha, \beta) : F(\alpha)][F(\alpha) : F]] = p^2.\]
现在,对于 $F(\alpha, \beta)$ 中的 $\gamma$,存在 $c_{ij} \in F$,使得 $\gamma = \sum c_{ij} \alpha^i \beta^j$。因此,
\[\gamma^p = \sum c_{ij}^p (\alpha^i \beta^j)^p = \sum c_{ij}^p Y^i Z^j \in F.\]
因此,$[F(\gamma) : F] \le p$,这意味着 $F(\alpha, \beta)$ 不能是 $F$ 的简单扩张。
\newpage
\section{Algebra Q31}
\section*{Exercise}
问题30. (Hungerford V.5.9) 如果 $n \ge 3$,证明 $x^{2^n} + x + 1$ 在 $\mathbb{F}_2$ 上是可约的。



\subsection*{Solution}
解:令 $p(x) = x^2 + x + 1$。假设,为了矛盾,$p(x)$ 是不可约的。令 $r$ 为 $p(x)$ 在某个分裂域 $F$ 中的根。则 $[\mathbb{F}_2(r) : \mathbb{F}_2] = n$。这意味着 $\mathbb{F}_2(r) = \mathbb{F}_{2^n}$。此外,我们知道 $\mathbb{F}_{2^n}$ 是多项式 $q(x) = x^{2^n} - x \in \mathbb{F}_2[x]$ 的分裂域。注意,对于任何 $a \in \mathbb{F}_{2^n}$,
\[ p(r+a) = (r+a)^{2^n} + (r+a) + 1 = (r^{2^n} + r + 1) + (a^{2^n} - a) = 0。 \]
因此,对于每个 $a \in \mathbb{F}_{2^n}$,$r+a$ 是 $p(x)$ 的根。由于 $r^{2^n} + r + 1 = \mathbb{F}_{2^n}$,$\mathbb{F}_{2^n}$ 中的任何元素都是 $p(x)$ 的根。特别地,$0 = p(0) = 1$,这是一个矛盾。因此,$p(x)$ 是不可约的。
\newpage
\section{Algebra Q32}
\section*{Exercise}
31. (Hungerford V.5.12) 设 $p$ 为素数。证明,对于任意正整数 $n$,$\mathbb{F}_p[x]$ 中存在一个次数为 $n$ 的不可约多项式。



\subsection*{Solution}
解:我们知道,存在一个唯一的(同构意义下)阶为 $p^n$ 的域,记为 $F$。令 $F^\times$ 为 $F$ 的单位的乘法子群。由于 $F$ 是 $\mathbb{F}_p$ 上的向量空间,且 $|F| = p^n$,因此 $[F : \mathbb{F}_p] = n$。由于 $F^\times$ 是有限的,因此它一定是循环的。因此存在 $a \in F$,使得 $F^\times = \langle a \rangle$。因此,$F = \mathbb{F}_p(a)$。令 $f(x)$ 为 $a$ 的不可约多项式。则 $\deg f = [\mathbb{F}_p(a) : \mathbb{F}_p]$。因此,$f(x)$ 是一个次数为 $n$ 的不可约多项式。
\newpage
\section{Algebra Q33}
\section*{Exercise}
问题 32. (Lang VI.11) 如果多项式 $f(x)$ 的根为 $\alpha$,则 $1/\alpha$ 也是其根,则称 $f(x)$ 为可逆多项式。假设 $f(x)$ 的系数在复数域的实子域 $k$ 中。若 $f(x)$ 在 $k$ 上不可约且有一个非实根的绝对值为 $1$,证明 $f(x)$ 是偶数次的可逆多项式。



\subsection*{Solution}
解:设 $\beta$ 为 $f(x)$ 的一个非实根,且其绝对值为 $1$,设 $\alpha$ 为 $f(x)$ 的任意一个根。设 $K$ 为 $f(x)$ 在 $\mathbb{C}$ 中的分裂域,设 $G$ 为 $K/k$ 的 Galois 群。由于 $f(x)$ 在 $k$ 上不可约,$G$ 在 $f(x)$ 的根上作用是可传递的。因此存在 $\sigma \in G$,使得 $\sigma(\beta) = \alpha$。这意味着 $\sigma(\overline{\beta}) = \overline{\sigma(\beta)} = \overline{\alpha}$。由于 $\beta$ 是 $f(x)$ 的根,因此 $\alpha^{-1}$ 也是 $f(x)$ 的根;这是因为 $\sigma$ 把 $f(x)$ 的根进行了排列。因此,$f(x)$ 是一个回文多项式。

由于 $f(x)$ 是一个不可约多项式,且其系数域的特征为零,因此 $f(x)$ 是可分的。我们知道,$f(x)$ 的非实根成对出现。由于 $f(x)$ 是回文多项式,因此实根也成对出现。因此,$f(x)$ 在 $K$ 中有偶数个根。因此,$f(x)$ 的次数为偶数。
\newpage
\section{Algebra Q34}
\section*{Exercise}
第33题(Lang VI.12)求有理数域上的多项式$x^5 - 4x + 2$的Galois群。



\subsection*{Solution}
解决方案:(a)令 $p(x) = x^5 - 4x + 2$,$G$ 为 $p(x)$ 的伽罗瓦群。我们将 $G$ 视为 $S_5$ 的子群。根据 Eisenstein 准则,多项式 $p(x)$ 在 $\mathbb{Z}$ 上是不可约的。高斯引理则意味着 $p(x)$ 在有理数上是不可约的。因此,$G$ 包含一个长度为 5 的循环。由于 $p'(x) = 5x^4 - 4$ 只有两个实根,$p(x)$ 最多有三个实根。鉴于 $p(-\infty) = -\infty$、$p(0) = 2$、$p(1) = -1$ 和 $p(\infty) = \infty$,$p(x)$ 恰好有三个实根。由于 $p(x)$ 只有两个非实根,它们是共轭的,共轭自同构表示 $G$ 中的一个交换。由于 $G \le S_5$ 包含一个 5-循环和一个交换,$G = S_5$。
\newpage
\section{Algebra Q35}
\section*{Exercise}
问题34. (Lang VI 13) 找出多项式 $x^4 + 2x^2 + x + 3$ 在有理数域上的伽罗瓦群。



\subsection*{Solution}
解:设 $p(x) = x^4 + 2x^2 + x + 3$。对 2 取模后,我们得到 $p(x) = x^4 + x^2 + 1$。由于 $p(x)$ 在 $\mathbb{Z}_2$ 中没有根,如果它不可约,则必须分解为两个不可约多项式的乘积,每个多项式的次数为 2。但是,$\mathbb{Z}_2$ 上唯一的不可约多项式是 $X^2 + X + 1$,而 $(x^2 + x + 1)^2 = x^4 + x^2 + 1 \neq p(x)$。因此,$p(x)$ 在 $\mathbb{Z}_2$ 上是不可约的。由于每个有限域的有限扩张都是循环的,$p(x)$ 在 $\mathbb{Z}_2$ 上的伽罗瓦群包含一个 4 阶元素。因此,$G$ 包含一个 4 阶元素。

现在对 3 取模,我们得到 $p(x) = x(x^3 + 2x + 1)$。由于 $x^3 + 2x + 1$ 在 $\mathbb{Z}_3$ 中没有根,因此它在 $\mathbb{Z}_3$ 上是不可约的。因此,$p(x)$ 在 $\mathbb{Z}_3$ 上的伽罗瓦群包含一个 3 阶元素。因此,$G$ 包含一个 3 阶元素。

由于 $G$ 包含一个 3 阶元素和一个 4 阶元素,因此 $|G|$ 可以被 12 整除。由于 $G$ 与 $S_4$ 的子群同构,因此 $G$ 与 $A_4$ 或 $S_4$ 同构。由于 $A_4$ 中没有 4 阶元素,因此 $G$ 必须与 $S_4$ 同构。
\newpage
\section{Algebra Q36}
\section*{Exercise}
**问题 35.** 求多项式 $x^5 - 5$ 在 $\mathbb{Q}$ 上的伽罗瓦群。



\subsection*{Solution}
解决方案:令 $p(x) = x^5 - 5$。根据 Eisenstein 准则和 Gauss 引理,$p(x)$ 在 $\mathbb{Q}$ 上是不可约的。令 $a = \sqrt[5]{5}$,$\omega$ 为 5 次单位根的原始根。则 $p(x)$ 在 $\mathbb{Q}$ 上的分裂域为 $F = \mathbb{Q}(a,\omega)$;这是因为 $p(x)$ 的根由 $\omega^i a$ 给出,其中 $1 \le i \le 5$。由于 $a$ 是 $p(x)$ 的根,而 $p(x)$ 在 $\mathbb{Q}$ 上是不可约的,因此我们有 $[\mathbb{Q}(a): \mathbb{Q}] = 5$。另一方面,由于 $\omega$ 的不可约多项式是 $x^4+x^3+x^2+x+1$,因此 $[\mathbb{Q}(\omega) : \mathbb{Q}] = 4$。由于 $(4,5) = 1$,因此 $[F: \mathbb{Q}] = [\mathbb{Q}(a) : \mathbb{Q}][\mathbb{Q}(\omega) : \mathbb{Q}] = 20$。因此,$p(x)$ 的 Galois 群 $G = \text{Gal}(F/\mathbb{Q})$ 的阶为 20。

根据问题 1,$S_5$ 不包含阶为 20 的任何 Abelian 子群。因此,$G$ 不是 Abelian 的。根据问题 4,$G$ 必须与阶为 20 的二面体或两个非同构的半直接积 $\mathbb{Z}_5 \rtimes_\phi \mathbb{Z}_4$ 中的一个同构。
\newpage
\section{Algebra Q37}
\section*{Exercise}
求多项式 $x^5 - 5$ 在 $\mathbb{Q}$ 上的伽罗瓦群。



\subsection*{Solution}
设 $p(x) = x^5 - 5$。根据 Eisenstein 判别法和高斯引理,$p(x)$ 在 $\mathbb{Q}$ 上是不可约的。设 $a = \sqrt[5]{5}$,$\omega$ 为 5 次本原单位根。则 $p(x)$ 在 $\mathbb{Q}$ 上的分裂域为 $F = \mathbb{Q}(a,\omega)$;这是因为 $p(x)$ 的根由 $a\omega^i$ 给出,其中 $1 \le i \le 5$。由于 $a$ 是 $p(x)$ 的根,而 $p(x)$ 在 $\mathbb{Q}$ 上是不可约的,因此我们有 $[\mathbb{Q}(a): \mathbb{Q}] = 5$。另一方面,由于 $\omega$ 的不可约多项式是 $x^4+x^3+x^2+x+1$,因此 $[\mathbb{Q}(\omega) : \mathbb{Q}] = 4$。由于 $(4,5) = 1$,因此 $[F: \mathbb{Q}] = [\mathbb{Q}(a) : \mathbb{Q}][\mathbb{Q}(\omega) : \mathbb{Q}] = 20$。因此,$p(x)$ 的 Galois 群 $G = \text{Gal}(F/\mathbb{Q})$ 的阶为 20。

根据问题 1,$S_5$ 中没有阶为 20 的阿贝尔子群。因此,$G$ 不是阿贝尔群。根据问题 4,$G$ 必须与阶为 20 的二面体群或两个非同构的半直接积 $\mathbb{Z}_5 \rtimes_\phi \mathbb{Z}_4$ 中的一个同构。
\newpage
\section{Algebra Q38}
\section*{Exercise}
问题36.(Lang VI.14)证明,对于一个对称群$S_n$,存在一个多项式$f(x) \in \mathbb{Z}[x]$,其首项系数为1,其在$\mathbb{Q}$上的伽罗瓦群是$S_n$。



\subsection*{Solution}
解决方案:我们在问题 30 中看到,对于任何素数 $p$ 和任何 $n \in \mathbb{N}$,存在一个不可约多项式 $g(x) \in \mathbb{F}_p[x]$ 的次数为 $n$。取一个不可约多项式 $p_1(x) \in \mathbb{F}_2[x]$ 的次数为 $n$。另外,取一个不可约多项式 $p_2(x) \in \mathbb{F}_3[x]$ 的次数为 $n-1$。最后,取一个不可约多项式 $p_3(x) \in \mathbb{F}_5[x]$ 的次数为 $n$。根据中国剩余定理,存在一个多项式 $f(x) \in \mathbb{Z}[x]$,使得
\[
\begin{array}{rcll}
f(x) & = & p_1(x) & (\text{mod } 2) \\
f(x) & = & xp_2(x) & (\text{mod } 3) \\
f(x) & = & q(x)p_3(x) & (\text{mod } 5)
\end{array}
\]
其中 $q(x)$ 是奇数次数不可约多项式的乘积,选择方便。设 $G$ 为 $f(x)$ 在有理数域上的 Galois 群,视为 $S_n$ 的子群。第一个等式意味着 $G$ 包含一个 $n$ 个循环,因此 $G$ 是 $S_n$ 的一个超越子群。第二个和第三个等式分别保证存在一个 $(n-1)$ 个循环和一个置换在 $G$ 中。正如我们在前一个问题中看到的,如果一个超越子群 $S_n$ 包含一个 $(n-1)$ 个循环和一个置换,那么它一定是全群。因此,$G \cong S_n$。
\newpage
\section{Algebra Q39}
\section*{Exercise}
问题37. (Lang VI.23) 证明以下陈述。
(a) 设$G$为一个阿贝尔群。则存在一个阿贝尔扩张,其Galois群为$G$。
(b) 设$k$为$\mathbb{Q}$的有限扩张,$G \neq \{1\}$为一个有限阿贝尔群。则存在无穷多个阿贝尔扩张,其Galois群为$G$。



\subsection*{Solution}
(a) 根据有限生成的阿贝尔群的基本定理,我们有 $G \cong \mathbb{Z}_{n_1} \times \dots \times \mathbb{Z}_{n_k}$,其中 $n_1, \dots, n_k \in \mathbb{N}$。根据狄利克雷定理,对于 $i \in \{1, \dots, k\}$,存在无穷多个素数 $p$,使得 $p-1 \in (n_i)$。因此,我们可以选择不同的素数 $p_1, \dots, p_k$,使得 $p_i - 1 \in (n_i)$。由于 $(p_i)$ 和 $(p_j)$ 是互质的,对于 $i \neq j$,如果 $n = p_1 \dots p_k$,根据中国剩余定理,我们有 $\mathbb{Z}_n \cong \mathbb{Z}_{p_1} \times \dots \times \mathbb{Z}_{p_k}$。这意味着
\[ (\mathbb{Z}_n)^\times \cong (\mathbb{Z}_{p_1})^\times \times \dots \times (\mathbb{Z}_{p_k})^\times。 \]
因此,我们有 $(\mathbb{Z}_n)^\times \cong \mathbb{Z}_{p_1-1} \times \dots \times \mathbb{Z}_{p_k-1}$。
现在,如果 $\zeta$ 是一个原始的 $n$ 次单位根,
\[ H = \text{Gal}(\mathbb{Q}(\zeta)/\mathbb{Q}) \cong (\mathbb{Z}_n)^\times \cong \mathbb{Z}_{p_1-1} \times \dots \times \mathbb{Z}_{p_k-1}。 \]
注意到 $H$ 有一个子群 $N = N_1 \times \dots \times N_k$,其中 $N_i$ 是 $\mathbb{Z}_{p_i-1}$ 的一个循环子群,阶为 $n_i$。由于 $N$ 是阿贝尔群,$N$ 是 $H$ 的一个正常子群。设 $F$ 为 $N$ 在伽罗瓦扩张 $\mathbb{Q}(\zeta)/\mathbb{Q}$ 中的不动点域。然后,根据伽罗瓦对应定理,$F/\mathbb{Q}$ 是伽罗瓦扩张,其伽罗瓦群由 $H/N$ 给出。由于 $\mathbb{Z}_{p_i-1}/N_i \cong \mathbb{Z}_{n_i}$,我们有 $H/N \cong \mathbb{Z}_{n_1} \times \dots \times \mathbb{Z}_{n_k} = G$。因此,我们得到所需的结果。
(b) 根据狄利克雷定理,存在无穷多个素数 $p$,使得 $p-1 \in (n_i)$。因此,我们可以创建一个族 $\mathcal{F} = \{S_i : i \in \mathbb{N}\}$,其中 $S_i = \{p_{i1}, \dots, p_{ik}\}$,使得 $p_{ij}$ 是素数,满足 $p_{ij} - 1 \in (n_j)$,且 $S_r \cap S_t$ 为空集,对于 $r \neq t$。定义 $c_i = \prod_{j=1}^k p_{ij}$,对于所有 $i \in \mathbb{N}$。现在考虑扩张 $\mathbb{Q}(\zeta_{c_i})$,其中 $\zeta_{c_i}$ 是一个原始的 $c_i$ 次单位根。由于 $(c_r, c_t) = 1$,对于 $r \neq t$,我们有 $\mathbb{Q}(\zeta_{c_r}) \cap \mathbb{Q}(\zeta_{c_t}) = \mathbb{Q}$。对于每个 $i$,我们生成一个中间域 $F_i$,其伽罗瓦群为 $G$,与我们在 (a) 中所做的类似。由于 $[k : \mathbb{Q}] < \infty$ 且 $\mathbb{Q}(\zeta_{c_r}) \cap \mathbb{Q}(\zeta_{c_t}) = \mathbb{Q}$,对于所有 $r \neq t$,只有有限多个 $i$ 使得 $\mathbb{Q}(\zeta_{c_i}) \cap k$ 严格包含 $\mathbb{Q}$。因此,对于无穷多个 $i$,满足 $F_i \cap k = \mathbb{Q}$,
\[ \text{Gal}(kF_i/k) \cong \text{Gal}(F_i/\mathbb{Q}) \cong G。 \]
\newpage
\section{Algebra Q40}
\section*{Exercise}
问题38. (Lang VI.24) 证明存在无穷多个非零互质整数 $a, b$,使得 $-4a^3 - 27b^2$ 是 $\mathbb{Z}$ 中的完全平方数。



\subsection*{Solution}
解决方案:(感谢我的教授Vera Serganova提供的解决方案。)我们可以通过以下方式来实现。我们想要 $d^2 = -4a^3 - 27b^2$ 或等价地,$a^3 = (d^2 + 27b^2)/4$。我们注意到右边是域 $\mathbb{Q}(\omega)$ 中 $(d+3b\sqrt{-3})/2$ 的范数,其中 $\omega$ 是一个原始的三次方根。对于任何 $\alpha \in \mathbb{Z}[\omega]$,$\alpha^3$ 的范数是 $\alpha$ 的范数的立方。由于 $\alpha$ 的范数是整数,我们可以取任何 $\alpha$ 并设置 $\alpha^3 = (d+3b\sqrt{-3})/2$,然后 $a$ 就是 $\alpha$ 的范数。为了使 $a$ 和 $b$ 相对质,我们可以取例如 $a = (1 + 3p\sqrt{-3})/2$,其中 $p$ 是素数。然后 $b$ 和 $d$ 是互质的,因此 $a$ 和 $b$ 也互质。
\newpage
\section{Algebra Q41}
\section*{Exercise}
问题39. (Lang VI.31) 设$F$为有限域,$K$为$F$的有限扩张。证明$N_{K/F}$和$T_{K/F}$是满射(作为从$K$到$F$的映射)。



\subsection*{Solution}
证明:设 $p$ 为 $F$ 的特征。我们将 $T_{K/F}$ 和 $N_{K/F}$ 简写为 $T$ 和 $N$。首先,我们证明迹是满射的。由于 $T: K \to F$ 是 $F$ 上的向量空间的线性变换,且 $F$ 的维数为 1,因此 $\text{Im}(T)$ 要么是 $0$,要么是 $F$。由于 $K/F$ 是有限伽罗瓦扩张,其伽罗瓦群 $G$ 是有限的。因此,根据 Artin 定理,$G$ 的元素必须线性独立。这意味着存在 $a \in K^\times$ 使得 $T(a) \neq 0$。因此 $\text{Im}(T) = F$。现在我们证明 $N$ 是满射。假设 $|K| = p^n$。由于 $K/F$ 是有限且可分的,因此扩张 $K/F$ 是伽罗瓦扩张。设 $G$ 为 $K/F$ 的伽罗瓦群。由于每个有限域的有限扩张都是循环的,因此存在 $\varphi \in G$ 使得 $G = \langle \varphi \rangle$。另一方面,$K$ 和 $F$ 的单位群 $K^*$ 和 $F^*$ 是循环的。由于 $N$ 是乘法的,因此它诱导了一个群同态 $L_N: K^* \to F^*$,由 $L_N(a) = N(a)$ 给出。$K^*$ 中的元素 $a$ 在 $\ker(L_N)$ 中当且仅当 $$1 = \prod_{i=0}^{n-1} \varphi^i(a) = \prod_{i=0}^{n-1} a^{p^i} = a^{1 + p + \cdots + p^{n-1}}。$$ 这当且仅当 $a$ 是多项式 $p(x) = x^c - 1$ 的根,其中 $c = 1 + p + \cdots + p^{n-1}$。因此 $|\ker(L_N)| = c$,因此 $$|K^*/\ker(L_N)| = \frac{p^n - 1}{c} = p - 1。$$ 根据第一个同构定理,$|\text{Im}(L_N)| = p - 1$。因此,$L_N$ 是满射的,然后 $N$ 也满射。
\newpage
\section{Algebra Q42}
\section*{Exercise}
问题40. (Hungerford V.8.9) 如果 $n > 2$ 且 $\zeta$ 是 $\mathbb{Q}$ 上的原始 $n$ 次单位根,则 $[\mathbb{Q}(\zeta + \zeta^{-1}) : \mathbb{Q}] = \phi(n)/2$。



\subsection*{Solution}
解:设 $K$ 为域 $\mathbb{Q}(\zeta+\zeta^{-1})$。设 $G$ 为域扩张 $\mathbb{Q}(\zeta)/\mathbb{Q}$ 的伽罗瓦群。设 $\sigma \in G$ 为共轭自同构。考虑 $G$ 的循环子群 $\langle \sigma \rangle$。我们将证明 $K$ 是 $\langle \sigma \rangle$ 的不动点域。很容易看出 $\sigma$ 固定 $K$。假设 $\phi \in G$ 固定 $\zeta + \zeta^{-1}$。则
\[\zeta + \zeta^{-1} = \phi(\zeta + \zeta^{-1}) = \phi(\zeta) + \phi(\zeta)^{-1}.\]
由于 $\zeta$ 是原始的,$\phi(\zeta) = \zeta^i$,其中 $i$ 为某个整数。将 $\phi(\zeta) = \zeta^i$ 代入上式,我们得到 $(\zeta+\zeta^{-1} - 1)(\zeta^{i-1} - 1) = 0$。因此 $i$ 要么是 $1 \pmod{n}$,要么是 $-1 \pmod{n}$。因此,$K$ 是 $\langle \sigma \rangle$ 的不动点域。然后,由伽罗瓦对应定理可知,$[\mathbb{Q}(\zeta): K] = 2$。这意味着
\[[K : \mathbb{Q}] = [\mathbb{Q}(\zeta) : \mathbb{Q}]/[\mathbb{Q}(\zeta) : K] = \phi(n)/2.\]
\newpage
\section{Algebra Q43}
\section*{Exercise}
41. (Hungerford V.8.9) 设 $p$ 为素数,$\zeta$ 为 $p$ 次本原单位根。求出所有满足 $[F : \mathbb{Q}] = 2$ 的 $\mathbb{Q}(\zeta)$ 的子域 $F \subseteq \mathbb{Q}(\zeta)$。



\subsection*{Solution}
解决方案:扩张 $\mathbb{Q}(\zeta)/\mathbb{Q}$ 的伽罗瓦群 $G$ 与 $(\mathbb{Z}/p\mathbb{Z})^*$ 同构。由于 $G$ 是循环群且 $|G| = p-1$,因此它仅包含一个阶为 $(p-1)/2$ 的子群 $H$。这意味着,根据伽罗瓦对应定理,存在唯一一个中间域 $F$,使得 $\mathbb{Q}(\zeta)/\mathbb{Q}$ 的度为 $2$,即 $H$ 的不动点域。由于 $H$ 是一个循环群,我们可以写成 $H = \langle \sigma \rangle$,其中 $\sigma \in H$。如果 $\tau = T_{\mathbb{Q}(\zeta)/\mathbb{Q}}(x) \in F$,则 $\mathbb{Q}(\zeta)/\mathbb{Q}$ 不一定是 $p$ 的素数。由于 $\zeta - \zeta^{-1}$ 是 $\mathbb{Q}(\zeta)$ 的形式,因此可以将其唯一地写为 $\zeta^i$ 的线性组合。事实上,$G$ 固定 $F$,这意味着对于某个 $x$,$\sigma(x) = x$。因此,$p = \sigma^j \in H$。因此,我们可以得出结论:$F = \mathbb{Q}(\tau)$ 是 $\mathbb{Q}$ 上唯一的中间域,其度为 $2$。
\newpage
\section{Algebra Q44}
\section*{Exercise}
证明任何有限群都与某个有限扩张 $F \subseteq E$ 的伽罗瓦群同构。



\subsection*{Solution}
假设 $G$ 是一个有限群,阶为 $n$。$G$ 在自身上的左乘法作用诱导了一个同态 $f: G \to S_n$。由于 $f$ 是单射,我们可以把 $G$ 想象成 $S_n$ 的一个子群。此外,我们已经看到,对于每个 $n$,对称群 $S_n$ 都是域扩张 $E/F$ 的伽罗瓦群。由于 $G$ 是 $S_n$ 的一个子群,根据伽罗瓦对应关系,在扩张 $E/F$ 中存在一个中间域 $K$,使得 $\text{Gal}(E/K)$ 与 $G$ 同构。
\newpage
\section{Algebra Q45}
\section*{Exercise}
问题 43. 设 $\overline{\mathbb{Q}} \subset \mathbb{C}$ 表示代数数的子域,$G$ 表示 $\overline{\mathbb{Q}}$ 在 $\mathbb{Q}$ 上的(无穷)伽罗瓦群。我们称 $\alpha \in \overline{\mathbb{Q}}$ 为完全实数,如果对于任何 $g \in G$,$g(\alpha) \in \mathbb{R}$。
(a) 证明所有完全实数的集合 $H$ 是 $\overline{\mathbb{Q}}$ 的一个子域。
(b) 域扩张 $H/\mathbb{Q}$ 是否是正常的?



\subsection*{Solution}
(a) 假设 $\alpha, \beta \in H$。对于任何 $g \in G$,$g(0) = 0 \in \mathbb{R}$ 且 $g(\alpha + \beta) = g(\alpha) + g(\beta) \in \mathbb{R}$。此外,$g(1) = 1$,如果 $\beta \neq 0$,则 $g(\alpha\beta^{-1}) = g(\alpha)g(\beta)^{-1} \in \mathbb{R}$。因此,$H$ 是 $\overline{\mathbb{Q}}$ 的一个子域。

(b) 假设 $f(x) \in \mathbb{Q}[x]$ 是一个不可约多项式,其分裂域为 $F \subset \overline{\mathbb{Q}}$。设 $\alpha$ 为 $f(x)$ 在 $H$ 中的一个根。设 $\beta \in F$ 为 $f(x)$ 的另一个根。由于 $f(x)$ 是不可约的,$G_F = \text{Gal}(F/\mathbb{Q})$ 在 $f(x)$ 的根上作用是可交换的。因此存在 $\sigma \in G_F$,使得 $\sigma(\alpha) = \beta$。由于 $F$ 的任何自同构都可以扩展到其代数闭包 $\overline{\mathbb{Q}}$,因此存在 $\overline{\sigma} \in G$,使得 $\overline{\sigma}|_F = \sigma$。现在假设 $g$ 是 $G$ 的任意元素。则 $g(\beta) = (g \circ \overline{\sigma})(\alpha) \in \mathbb{R}$;这是因为 $g \circ \overline{\sigma} \in G$ 且 $\alpha \in H$。因此,$\beta \in H$。因此,$f(x)$ 的所有根都在 $H$ 中。由于 $f(x)$ 是任意选择的,因此 $H/\mathbb{Q}$ 是一个正规扩张。
\newpage
\section{Algebra Q46}
\section*{Exercise}
问题 44. 设 $p$ 为素数,$F$ 为多项式 $x^{p^r}-1$ 的分裂域,其中 $r > 0$。证明 $F$ 在 $\mathbb{Q}$ 上的伽罗瓦群与逆极限 $\varprojlim (\mathbb{Z}/p^r\mathbb{Z})^*$ 同构。



\subsection*{Solution}
解:多项式 $x^{p^r}-1$ 的分裂域是 $F_r = \mathbb{Q}(\zeta_{p^r})$,其中 $\zeta_{p^r}$ 是 $p^r$ 次单位根的原始根。因此,$F = \mathbb{Q}(\zeta_p, \zeta_{p^2}, \dots)$。由于每个 $\zeta_{p^r}$ 都是 $\mathbb{Q}$ 上的可分扩张,因此 $F$ 也可分。因此,$F/\mathbb{Q}$ 是一个伽罗瓦扩张。设 $G$ 为扩张 $F/\mathbb{Q}$ 的伽罗瓦群,$G_r \cong (\mathbb{Z}/p^r\mathbb{Z})^*$ 为扩张 $F_r/\mathbb{Q}$ 的伽罗瓦群。

对于 $j \ge i$,我们定义 $\phi_{j,i}: G_j \to G_i$ 为 $\sigma \mapsto \sigma|_{F_i}$。则 $(G_r, \phi_{j,i})$ 是一个有向的群族。设 $G$ 为其逆极限。我们证明 $G$ 与 $G$ 同构。定义映射 $f: G \to G$ 为 $f(\sigma) = (\sigma|_{F_r})$。由于 $\sigma|_{F_r}(\zeta_{p^r}) = \zeta_{p^r}^k$,其中 $k \in \mathbb{Z}$,因此映射 $f$ 是良定义的。同时,$f$ 也是一个同态。我们证明 $f$ 是单射的。如果 $f(\sigma) = f(\tau)$,则对于任何 $r > 0$,$\sigma|_{F_r}$ 都是恒等映射。由于 $F = \cup F_r$,对于每个 $x \in F$,都存在 $r$ 使得 $x \in F_r$。因此,$\sigma(x) = \sigma|_{F_r}(x) = x$。因此,$f$ 是单射的。现在,取 $(s_r) \in G$。定义 $\sigma \in G$ 如下:对于 $x \in F$,取 $r$ 使得 $x \in F_r$。然后设 $\sigma(x) = s_r(x)$。假设 $x \in F_r$ 且 $x \in F_s$,其中 $r  e s$。由于 $(s_r)$ 在 $G$ 中,因此 $s_r|_{F_s} = s_s$。因此,$\sigma(x)$ 不依赖于 $r$ 的选择。由于 $F_1 \subset F_2 \subset \dots$ 是一个递增的域序列,其并集是 $F$,因此对于每个 $r > 0$,$F_r/\mathbb{Q}$ 都是自同构,这意味着 $\sigma$ 是 $F$ 的自同构。由于 $\sigma \in G$ 且 $f(\sigma) = (\sigma|_{F_r})$,因此同态 $f$ 是满射的。因此,$f$ 是一个同构,这意味着

\[
\text{Gal}(F/\mathbb{Q}) = G \cong \varprojlim_{r} \text{Gal}(F_r/\mathbb{Q}) \cong \varprojlim_{r} (\mathbb{Z}/p^r\mathbb{Z})^*.
\]
\newpage
\section{Algebra Q47}
\section*{Exercise}
问题 45. 证明以下陈述。证明 $\text{PSL}(2, \mathbb{F}_5) \approx A_5$。





\subsection*{Solution (by LLM)}
由LLM生成的解答可能不准确,请自行验证。

\textbf{证明:} 我们需要证明$\text{PSL}(2, \mathbb{F}_5) \approx A_5$,即$\text{PSL}(2, \mathbb{F}_5)$与$A_5$同构。

首先,$\text{PSL}(2, \mathbb{F}_5)$是$\text{SL}(2, \mathbb{F}_5)$模去其中心$\{I, -I\}$得到的群,其中$I$是单位矩阵。$\text{SL}(2, \mathbb{F}_5)$是所有行列式为1的$2 \times 2$矩阵构成的群,$\mathbb{F}_5$是5个元素的有限域。

$\text{PSL}(2, \mathbb{F}_5)$的阶数为:
\[
|\text{PSL}(2, \mathbb{F}_5)| = \frac{|\text{SL}(2, \mathbb{F}_5)|}{|\{I, -I\}|} = \frac{120}{2} = 60
\]
而$A_5$是5个元素的交错群,其阶数也是60。

接下来,我们证明$\text{PSL}(2, \mathbb{F}_5)$与$A_5$同构。首先,$\text{PSL}(2, \mathbb{F}_5)$是一个单群,因为$\text{SL}(2, \mathbb{F}_5)$是一个单群,且$\text{PSL}(2, \mathbb{F}_5)$是其商群。$A_5$也是一个单群。

由于$\text{PSL}(2, \mathbb{F}_5)$和$A_5$都是阶数为60的单群,且在同构意义下,阶数为60的单群只有$A_5$,因此$\text{PSL}(2, \mathbb{F}_5) \approx A_5$。

综上所述,我们证明了$\text{PSL}(2, \mathbb{F}_5) \approx A_5$。
\newpage
\section{Algebra Q48}
\section*{Exercise}
问题 46. 将所有的 $\mathbb{Z}$ 理想视为一个有向系统,通过可整除性。证明 $$\lim_{\leftarrow} \mathbb{Z}/(a) = \prod_{p} \mathbb{Z}_{p},$$ 其中极限是取所有理想 $(a)$,并且乘积是取所有素数 $p$。





\subsection*{Solution (by LLM)}
由LLM生成的解答可能不准确,请自行验证。

\textbf{解答:}

我们需要证明在由整除关系构成的有向系统中,$\mathbb{Z}$的所有理想的逆向极限同构于所有$p$-进整数环的直积,即
\[
\lim_{\leftarrow} \mathbb{Z}/(a) \cong \prod_{p} \mathbb{Z}_{p}.
\]

\textbf{步骤1:理解逆向极限的定义}

给定由所有非零理想$(a)$构成的有向集,其中$(a) \leq (b)$当且仅当$b \mid a$。对于每对$(a) \leq (b)$,存在自然的投影同态$\pi_{b,a}: \mathbb{Z}/(b) \to \mathbb{Z}/(a)$,定义为$x + (b) \mapsto x + (a)$。逆向极限$\lim_{\leftarrow} \mathbb{Z}/(a)$是所有相容族$(x_a)_{a \in \mathbb{Z}\setminus\{0\}}$的集合,其中$x_a \in \mathbb{Z}/(a)$,且对于$(a) \leq (b)$,有$\pi_{b,a}(x_b) = x_a$。

\textbf{步骤2:中国剩余定理的应用}

对于任意正整数$a$,设其质因数分解为$a = \prod_{p} p^{n_p}$,其中$n_p \geq 0$且只有有限多个$n_p > 0$。根据中国剩余定理,有
\[
\mathbb{Z}/(a) \cong \prod_{p} \mathbb{Z}/(p^{n_p}).
\]
因此,逆向极限可以分解为
\[
\lim_{\leftarrow} \mathbb{Z}/(a) \cong \lim_{\leftarrow} \prod_{p} \mathbb{Z}/(p^{n_p}).
\]

\textbf{步骤3:分解为各质数的逆向极限}

由于逆向极限与直积可交换,我们有
\[
\lim_{\leftarrow} \prod_{p} \mathbb{Z}/(p^{n_p}) \cong \prod_{p} \lim_{\leftarrow} \mathbb{Z}/(p^{n_p}).
\]
对于每个固定的质数$p$,$\mathbb{Z}/(p^{n_p})$的逆向极限$\lim_{\leftarrow} \mathbb{Z}/(p^{n_p})$正是$p$-进整数环$\mathbb{Z}_p$的定义。因此,
\[
\prod_{p} \lim_{\leftarrow} \mathbb{Z}/(p^{n_p}) = \prod_{p} \mathbb{Z}_p.
\]

\textbf{步骤4:验证同构}

综上,我们得到
\[
\lim_{\leftarrow} \mathbb{Z}/(a) \cong \prod_{p} \mathbb{Z}_p.
\]
这一同构将每个相容族$(x_a)_a$映射到其在各质数$p$处的投影$(x_{p^{n_p}})_p$,其中$x_{p^{n_p}} \in \mathbb{Z}_p$。

\textbf{结论:}

因此,原命题得证:
\[
\lim_{\leftarrow} \mathbb{Z}/(a) \cong \prod_{p} \mathbb{Z}_{p}.
\]
\newpage
\section{Algebra Q49}
\section*{Exercise}
问题47. 找到一个环$R$,使得$R$与$R^{\text{op}}$不全同。





\subsection*{Solution (by LLM)}
由LLM生成的解答可能不准确,请自行验证。

\textbf{问题47.} 找到一个环$R$,使得$R$不同构于$R^{\text{op}}$。

\textbf{解答:}

我们考虑四元数环$\mathbb{H}$。四元数环$\mathbb{H}$由形如$a + bi + cj + dk$的元素组成,其中$a, b, c, d \in \mathbb{R}$,且$i, j, k$满足以下乘法关系:
\[
i^2 = j^2 = k^2 = -1, \quad ij = k, \quad jk = i, \quad ki = j, \quad ji = -k, \quad kj = -i, \quad ik = -j.
\]

四元数环$\mathbb{H}$的乘法是非交换的,即存在$q_1, q_2 \in \mathbb{H}$使得$q_1 q_2  eq q_2 q_1$。因此,$\mathbb{H}$与它的对偶环$\mathbb{H}^{\text{op}}$不同构。

具体来说,$\mathbb{H}^{\text{op}}$的乘法定义为:
\[
q_1 \cdot_{\text{op}} q_2 = q_2 q_1.
\]
由于$\mathbb{H}$的乘法是非交换的,$\mathbb{H}$与$\mathbb{H}^{\text{op}}$的乘法结构不同,因此$\mathbb{H}$不同构于$\mathbb{H}^{\text{op}}$。

综上所述,四元数环$\mathbb{H}$是一个满足$R$不同构于$R^{\text{op}}$的环。
\newpage
\section{Algebra Q51}
\section*{Exercise}
问题49. 设k为一个域,G为一个有限群,k[G]表示群环。
(a) 证明:任何一个有限生成的k[G]-模块都是k上的有限维模块。
(b) 证明:任何一个有限维的射影模块都是注射模块。





\subsection*{Solution (by LLM)}
由LLM生成的解答可能不准确,请自行验证。

\textbf{Problem 49.} Let $k$ be a field, $G$ be a finite group, and $k[G]$ denote the group ring.

\textbf{(a)} Show that any finitely generated $k[G]$-module is finite-dimensional over $k$.

\textbf{Solution (a):} Let $M$ be a finitely generated $k[G]$-module. Then there exists a finite set $\{m_1, \dots, m_n\}$ such that $M = k[G]m_1 + \cdots + k[G]m_n$. Since $G$ is finite, the group ring $k[G]$ is finite-dimensional over $k$, with dimension $|G|$. 

For each $m_i$, the cyclic submodule $k[G]m_i$ is a quotient of $k[G]$ by the annihilator of $m_i$, hence $\dim_k(k[G]m_i) \leq \dim_k(k[G]) = |G|$. Therefore, $M$ is a sum of $n$ subspaces each of dimension at most $|G|$, so $\dim_k(M) \leq n|G| < \infty$.

\textbf{(b)} Show that any finite dimensional projective module is injective.

\textbf{Solution (b):} Let $P$ be a finite-dimensional projective $k[G]$-module. Since $k[G]$ is a finite-dimensional algebra over $k$, it is a Frobenius algebra. In particular, $k[G]$ is self-injective, meaning that $k[G]$ is injective as a $k[G]$-module. 

Since $P$ is projective, it is a direct summand of a free module $k[G]^n$ for some $n$. But $k[G]^n$ is injective because $k[G]$ is injective and finite direct sums of injective modules are injective. Therefore, $P$ is a direct summand of an injective module, hence injective itself. 

This completes the proof.
\newpage
\section{Algebra Q52}
\section*{Exercise}
问题50. (Lang III.25) 证明任意模块都是有限呈现模块的有向极限。





\subsection*{Solution (by LLM)}
由LLM生成的解答可能不准确,请自行验证。

\textbf{Problem 50. (Lang III.25)} Show that any module is a directed limit of finite presented modules.

\textbf{Solution:}

Let $M$ be a module over a ring $R$. We will construct $M$ as a directed limit of finitely presented modules.

 oindent\textbf{Step 1: Construct the directed system.}

Consider the set $\mathcal{F}$ of all finitely generated submodules of $M$. For $N_1, N_2 \in \mathcal{F}$, define $N_1 \leq N_2$ if $N_1 \subseteq N_2$. This makes $\mathcal{F}$ a directed set because for any $N_1, N_2 \in \mathcal{F}$, $N_1 + N_2 \in \mathcal{F}$ and $N_1, N_2 \leq N_1 + N_2$.

For each $N \in \mathcal{F}$, let $F_N$ be a finitely generated free module with a surjection $\pi_N: F_N \to N$. Let $K_N = \ker(\pi_N)$, so we have an exact sequence:
\[
0 \to K_N \to F_N \xrightarrow{\pi_N} N \to 0
\]
Since $N$ is finitely generated, $K_N$ is also finitely generated (as $R$ is Noetherian in many cases, but more generally, we can choose $F_N$ and $\pi_N$ such that $K_N$ is finitely generated). Thus, $N$ is finitely presented.

 oindent\textbf{Step 2: Define the transition maps.}

For $N_1 \leq N_2$ in $\mathcal{F}$, the inclusion $\iota_{N_1, N_2}: N_1 \to N_2$ induces a homomorphism $\phi_{N_1, N_2}: F_{N_1} \to F_{N_2}$ such that the following diagram commutes:
\[
\begin{array}{ccc}
F_{N_1} & \xrightarrow{\pi_{N_1}} & N_1 \\
\downarrow{\phi_{N_1, N_2}} & & \downarrow{\iota_{N_1, N_2}} \\
F_{N_2} & \xrightarrow{\pi_{N_2}} & N_2
\end{array}
\]
This gives a directed system $(F_N, \phi_{N_1, N_2})$ of finitely presented modules.

 oindent\textbf{Step 3: Take the directed limit.}

The directed limit $\varinjlim F_N$ satisfies the universal property of the colimit. There is a natural homomorphism:
\[
\varinjlim F_N \to M
\]
induced by the $\pi_N: F_N \to N \hookrightarrow M$. This map is surjective because every element of $M$ is contained in some finitely generated submodule $N \in \mathcal{F}$, and hence in the image of some $F_N$.

To show injectivity, suppose $x \in \varinjlim F_N$ maps to $0$ in $M$. Then $x$ is represented by some $y \in F_N$ for some $N \in \mathcal{F}$, and $\pi_N(y) = 0$ in $M$. But $\pi_N(y) = 0$ implies $y \in K_N$, so $x = 0$ in $\varinjlim F_N$.

Thus, $\varinjlim F_N \cong M$, showing that $M$ is a directed limit of finitely presented modules. $\square$
\newpage
\section{Algebra Q53}
\section*{Exercise}
问题 51. (Lang III.26) 设 $E$ 为一个环上的模块,$(M_i)$ 为一个有向模块族。如果 $E$ 是有限生成的,证明自然同态
\[
\varinjlim \operatorname{Hom}(E, M_i) \to \operatorname{Hom}(E, \varinjlim M_i)
\]
是单射的。如果 $E$ 是有限表示的,证明这个同态是同构的。





\subsection*{Solution (by LLM)}
由LLM生成的解答可能不准确,请自行验证。

\textbf{解答:}

首先,我们回顾一下题目中的概念和符号。设$E$是一个环上的模,$(M_i)$是一个有向的模族。我们需要证明当$E$是有限生成模时,自然同态
\[
\varinjlim \operatorname{Hom}(E, M_i) \to \operatorname{Hom}(E, \varinjlim M_i)
\]
是单射;当$E$是有限展示模时,这个同态是同构。

\textbf{第一部分:$E$是有限生成模时,同态是单射。}

设$E$是有限生成模,即存在有限个元素$e_1, \ldots, e_n$生成$E$。我们需要证明自然同态
\[
\phi: \varinjlim \operatorname{Hom}(E, M_i) \to \operatorname{Hom}(E, \varinjlim M_i)
\]
是单射。

设$f \in \varinjlim \operatorname{Hom}(E, M_i)$,且$\phi(f) = 0$。这意味着对于每个$e_j$,$f(e_j)$在$\varinjlim M_i$中为零。由于$\varinjlim M_i$是$M_i$的直极限,存在某个$i$使得$f(e_j)$在$M_i$中为零。由于$E$是有限生成的,我们可以找到一个共同的$i$使得$f(e_j) = 0$对所有$j$成立。因此,$f$在$M_i$中为零,从而$f$在$\varinjlim \operatorname{Hom}(E, M_i)$中为零。这表明$\phi$是单射。

\textbf{第二部分:$E$是有限展示模时,同态是同构。}

设$E$是有限展示模,即存在一个有限生成自由模$F$和一个有限生成子模$K$使得$E \cong F/K$。我们需要证明自然同态$\phi$是同构。

首先,我们已经知道$\phi$是单射。现在我们需要证明$\phi$是满射。

设$g \in \operatorname{Hom}(E, \varinjlim M_i)$。由于$E$是有限展示的,我们可以将$g$提升到$F$上的同态$g': F \to \varinjlim M_i$。由于$F$是有限生成的,存在某个$i$使得$g'$可以分解为$F \to M_i \to \varinjlim M_i$。因此,$g$可以表示为$\phi(f)$,其中$f$是某个$\operatorname{Hom}(E, M_i)$中的元素。这表明$\phi$是满射。

综上所述,当$E$是有限展示模时,$\phi$是同构。

\textbf{结论:}

当$E$是有限生成模时,自然同态$\phi$是单射;当$E$是有限展示模时,$\phi$是同构。
\newpage
\chapter{Test Img 1}
\section*{Exercise}
整环$R$为UFD 当且仅当 每个非零素理想$0 \neq p \in Spec(R)$ 均包含素元.



\subsection*{Solution}
若$R$为UFD, 令$0 \neq p \in Spec(R)$,则存在$0 \neq a = \pi_1 ... \pi_n \in p$,其中$\pi_i$为素元. 因$p$为素理想,某一个$\pi_i$包含于$p$内.
若每个非零素理想均包含素元,假设非单位$x$无素因子分解,则根据引理3,主理想$(x)$中任何元素均无素因子分解.令$S$为$R$内一切素元生成的乘性子集,则根据上述性质3与引理3,$S^{-1}(x)$为$S^{-1}R$的真子理想. 于是 $S^{-1}R$ 有包含$x$的极大理想$m$. 但素理想$i^{-1}(m)$不含任何素元,矛盾.
\newpage
\chapter{Test Img 2}
\section*{Exercise}
含么交换环$R$有唯一素理想与$R$非单位必为幂零元素等价



\subsection*{Solution}
证明(1) $\Rightarrow$ (2)
假设 $R$ 有唯一素理想 $P$。由于所有素理想的交集是幂零元集合(即nilradical), 且 $P$ 是唯一素理想, 因此 nilradical 等于 $P$, 即 $P$ 由所有幂零元组成。
设 $a \in R$ 为非单位元素。由于 $a$ 不是单位, 它属于某个极大理想, 而极大理想是素理想, 因此 $a \in P$ (因为 $P$ 是唯一素理想)。故 $a$ 是幂零元。
因此, 每个非单位元素都是幂零元素。

证明(2) $\Rightarrow$ (1)
假设 $R$ 中每个非单位元素都是幂零元素。令 $N$ 为 nilradical, 即所有幂零元的集合。由于幂零元形成理想, $N$ 是理想。
注意到单位元素不可能是幂零元 (因为如果是单位且 $u^n=0$, 则 $1 = u^{-1}u$, 乘以 $u^{-n}$ 得 $1=0$, 矛盾), 因此 $N$ 恰好由所有非单位元素和零组成。
由于每个不在 $N$ 中的元素都是单位, $N$ 是极大理想 (因为任何真理想都包含在 $N$ 中)。
现在, 设 $P$ 为任意素理想。由于 $P$ 是真理想, $P \subseteq N$。另一方面, 素理想必须包含所有幂零元, 因此 $N \subseteq P$。故 $P = N$。
因此, 所有素理想都等于 $N$, 即 $R$ 有唯一素理想 $N$。
\newpage
\end{document}